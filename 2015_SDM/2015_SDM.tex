\documentclass{aiaa-tc}

\usepackage{color}
\usepackage{amsmath}
%\usepackage{overcite}
\usepackage{graphicx}
\usepackage{subfig}
\usepackage{authblk}
\usepackage{amsfonts}

\input basic.ltx
\def\directory{EPSF/}

%------------------------------------------------------------------------------
% MAS Additions
\newcommand{\mas}[1]{\textcolor{magenta}{#1}}
\newcommand{\qw}[1]{\textcolor{blue}{#1}}
%useful for showing deleted text
\renewcommand{\kill}[1]{\textcolor{red}{\sout{#1}}}                             

\usepackage[normalem]{ulem}  % mas addition
\newcommand{\uvec}[1]{\bar{#1}}
\newcommand{\tens}[1]{\underline{\underline{#1}}}
\renewcommand{\vec}[1]{\underline{#1}}
\renewcommand{\skew}[1]{\widetilde{#1}}
%------------------------------------------------------------------------------

\title{\kill{BeamDyn: an Efficient High-Fidelity Beam Solver in the FAST
Modularization Framework}
BeamDyn: A High-Fidelity Wind Turbine Blade Solver in the FAST
Modular Framework}


\author[1]{Qi Wang\thanks{Research Engineer, National Wind Technology Center, AIAA Senior Member. Email: Qi.Wang2@nrel.gov}}
\author[2]{Nick Johnson\thanks{Graduate Research Assistant, Department of
Mechanical Engineering.}}
 \author[1]{Michael A. Sprague\thanks{Senior Research Scientist, 
Computational Science Center.}}
 \author[1]{Jason Jonkman\thanks{Senior Engineer, National Wind Technology Center.}}
 \affil[1]{National Renewable Energy Laboratory, Golden, CO 80401}
 \affil[2]{Colorado School of Mines, Golden, CO 80401}
 
 \renewcommand\Authands{ and }

\begin{document}

\maketitle

\begin{abstract}
{BeamDyn, a Legendre-spectral-finite-element implementation of geometrically
exact beam theory (GEBT), is developed to meet the design challenges
associated with highly flexible composite wind turbine blades. In this
paper, the governing equations of GEBT are reformulated into a nonlinear
state-space form to support its coupling within the
\kill{modularization}\mas{modular} framework of the FAST wind turbine
computer-aided\mas{-}engineering (CAE) tool. Different time integration
schemes \mas{(implicit and explicit)} are implemented and examined for wind
turbine analysis.  Numerical examples and discussion on the efficiency and
accuracy of time and space discretization will be presented in the final
paper. A validation example of a realistic wind turbine blade will also be
presented.}     
\end{abstract}

\section{Introduction} Wind power installations in the U.S. have exceeded 60
GW, and have become an increasingly important part of the overall energy
portfolio. In recent years, the size of wind turbines has also increased in
the quest for economies of scale.  Larger wind turbine blades result in
structures that are highly flexible.  To ensure the performance and
reliability of wind turbines, it is crucial to make use of
computer-aided-engineering (CAE) tools that are capable of analyzing wind
turbine blades in an accurate and efficient manner. Although modern
computers \kill{make fully} \mas{enable} three-dimensional (3D) analysis
\mas{of a fully resolved blade}, \mas{such analyses are too
expensive for iterative design. More importantly, modern composite wind
turbine blades are very well suited to nonlinear beam models, which can
capture with high-fidelity the deformation response under realistic
operating conditions, and in a small fraction of the time required by a
fully resolved 3D simulation.} \kill{consuming to be used for effective
design-space exploration. Therefore it is preferable to have an efficient
yet accurate alternative.}

Beam models are widely used to analyze structures that have one of its
dimensions \mas{being} much larger than the other two.  Many engineering
structures are modeled as beams, e.g.,  bridges, joists, and helicopter
rotor blades.  Similarly, beam models are \kill{ideal} \mas{well suited} to
analyze, \mas{with high fidelity},  wind turbine blades, towers, and shafts.
Most wind turbine blades are constructed of composite materials, and
analysis of composite beams is more complicated than isotropic beams due to
the elastic coupling effects.  The geometrically exact beam theory (GEBT),
first proposed by Reissner\cite{Ressiner1973}, is a \kill{beam analysis
method} \mas{beam-deformation model} capable of \mas{enabling} efficient
analysis of highly flexible composite structures.  GEBT has demonstrated its
efficacy in helicopter rotor analysis.  Simo\cite{Simo1985} and Simo and
Vu-Quoc\cite{Simo1986} extended Reissner's initial work to include 3D
dynamic problems. Jeleni\'c and Crisfield\cite{Crisfield1999} derived a
finite-element (FE) method that interpolates the rotation field thereby
preserving the geometric exactness of this theory. It is noted that
Ibrahimbegovi\'c and his colleagues implemented this theory for
static\cite{Ibrahim1995} and dynamic\cite{Ibrahim1998} analysis. Readers are
referred to Hodges\cite{HodgesBeamBook}, where comprehensive derivations and
discussions on nonlinear composite-beam theories can be found. Recently, a
mixed formulation of GEBT along with the numerical implementation was
presented by Yu and Blair\cite{YuGEBT}.  

FAST is a CAE tool developed by the National Renewable Energy Laboratory
(NREL) for the purposes of wind turbine analysis for both land-based and
offshore wind turbines using realistic operating conditions.  The current
beam model in FAST is not capable of \kill{analyzing composite or highly
flexible wind turbine blades.} \mas{predictive analysis of highly flexible
composite wind turbine blades.}
Recently, FAST has been reformulated under a
new modularized framework that provides a rigorous means by which various
mathematical systems are implemented in distinct modules and are interconnected
to solve for the global, coupled, dynamic response of wind turbines and wind
plants \cite{Jonkman:2013,website:FASTModularizationFramework}.

In this paper, \kill{a first-order,} a three-dimensional displacement-based
implementation of the geometrically exact beam theory using Legendre
spectral finite elements is presented. The theory is reformulated in a
nonlinear state-space form for the purpose of integrating with the FAST
framework, thereby introducing an optional high-fidelity beam model as an
alternative to the current beam model. This work builds on previous efforts
that showed the implementation GEBT and spatial discretization executed
using Legendre spectral finite elements
(LSFEs)\cite{Wang:GEBT2013,Wang:SFE2013,Wang:GEBT2014,Sprague:FAST2014} for
analysis of composite wind turbine blades. The paper is organized as
follows.  First, the theoretical foundation of the geometrically exact beam
theory along with the reformulation of the governing equations into a
state-space form is introduced. Coupling to the FAST framework is then
discussed. Finally, \kill{verification} \mas{validation} examples are
provided to show the accuracy and efficiency of the present model for
composite wind turbine blades. 

~

\section{Geometrically Exact Beam Theory}

This section reviews the geometric exact beam theory for completeness of this paper. The content of this section can be found in many other papers and textbooks.
Figure~\ref{Kinematics} shows a beam in its initial undeformed
and deformed states. A reference frame $\mathbf{b}_i$ is introduced along the
beam axis for the undeformed state; a frame $\mathbf{B}_i$ is introduced
along each point of the deformed beam axis. Curvilinear coordinate $x_1$ defines the intrinsic parameterization of the reference line.
\begin{figure}
\centering
\includegraphics[width=5.0in]{\directory Kinematics.eps}
\caption{Schematic of beam deformation} \label{Kinematics}
\end{figure}
In this paper, we use matrix notation to denote vectorial or vectorial-like quantities. For example, we use a underline to denote a vector $\underline{u}$, a bar to denote unit vector $\bar{n}$, and double underline to denote a tensor $\underline{\underline{\Delta}}$. Note that sometimes the underlines only denote the dimension of the corresponding matrix. The governing equations of motion for geometric exact beam theory can be written as \cite{Bauchau:2010}
\begin{align}
	\label{GovernGEBT-1}
	\dot{\underline{h}} - \underline{F}^\prime &= \underline{f} \\
	\label{GovernGEBT-2}
	\dot{\underline{g}} + \dot{\tilde{u}} \underline{h} - \underline{M}^\prime + (\tilde{x}_0^\prime + \tilde{u}^\prime)^T \underline{F} &= \underline{m}
\end{align}
where $\vec{h}$ and $\vec{g}$ are the linear and angular momenta resolved in the inertial coordinate system, respectively; $\vec{F}$ and $\vec{M}$ are the beam's sectional forces and moments, respectively; $\vec{u}$ is the 1D displacement of the reference line; $\vec{x}_0$ is the position vector of a point along the beam's reference line; $\vec{f}$ and $\vec{m}$ are the distributed force and moment applied to the beam structure.  Notation $(\bullet)^\prime$ indicates a derivative with respect to the beam axis $x_1$ and $\dot{(\bullet)}$ indicates a derivative with respect to time. The tilde operator $(\skew{\bullet})$ defines a second-order, skew-symmetric tensor corresponding to the given vector. In the literature, it is also termed as "cross-product matrix". For example,
\[
	\skew{n} = 
	     		\begin{bmatrix}
			0 & -n_3 & n_2 \\
			n_3 & 0 & -n_1 \\
			-n_2 & n_1 & 0\\
			\end{bmatrix}	
\]
The constitutive equations relate the velocities to the momenta and the one-dimensional strain measures to the sectional resultants as
\begin{align}
	\label{ConstitutiveMass}
	\begin{Bmatrix}
	\underline{h} \\
	\underline{g}
	\end{Bmatrix}
	= \underline{\underline{\mathcal{M}}} \begin{Bmatrix}
	\dot{\underline{u}} \\
	\underline{\omega}
	\end{Bmatrix} \\
	\label{ConstitutiveStiff}
	\begin{Bmatrix}
	\underline{F} \\
	\underline{M}
	\end{Bmatrix}
	= \underline{\underline{\mathcal{C}}} \begin{Bmatrix}
	\underline{\epsilon} \\
	\underline{\kappa}
	\end{Bmatrix}
\end{align}
where $\underline{\underline{\mathcal{M}}}$ and
$\underline{\underline{\mathcal{C}}}$ are the $6 \times 6$ sectional mass
and stiffness matrices, respectively (note that they are not really tensors);
$\underline{\epsilon}$ and $\underline{\kappa}$ are the 1D strains and
curvatures, respectively. $\underline{\omega}$ is the angular velocity
vector that is defined by the rotation tensor $\underline{\underline{R}}$ as
$\underline{\omega} =
axial(\dot{\underline{\underline{R}}}~\underline{\underline{R}})$. The 1D
strain measures are defined as
\begin{equation}
    \label{1DStrain}
    \begin{Bmatrix}
        \vec{\epsilon} \\
        \vec{\kappa}
    \end{Bmatrix}
    =
    \begin{Bmatrix}
        \vec{x}^\prime_0 + \vec{u}^\prime - (\tens{R} ~\tens{R}_0) \bar{\imath}_1 \\
        \vec{k} + \tens{R}~ \vec{k}_i
    \end{Bmatrix}
\end{equation}
where $\vec{k} = \text{axial} (\tens{R}^\prime \tens{R}^T)$ is the sectional
curvature vector resolved in the inertial basis, $\vec{k}_i$ is the
corresponding initial curvature vector, and $\bar{\imath}_1$ is the unit
vector along $x_1$ direction in the inertial basis. It is noted that the
three sets of equations, including equations of motion
Eq.~\eqref{GovernGEBT-1} and \eqref{GovernGEBT-2}, constitutive equations
Eq.~\eqref{ConstitutiveMass} and \eqref{ConstitutiveStiff}, and kinematical
equations Eq.~\eqref{1DStrain}, provided a \kill{fully} \mas{full} mathematical description of elasticity problems. 

For a displacement-based finite element implementation, there are six
degree-of-freedoms at each node: three displacement components and three
rotation components. Here, we use $\vec{q}$ to denote the elemental
displacement array as $\underline{q}^T=\left[
\underline{u}^T~~\underline{p}^T\right]$ where $\vec{u}$ is the
displacement and $\vec{p}$ is the rotation-parameter vector. The
acceleration array can thus be defined as $\underline{a}^T=\left[
\ddot{\underline{u}}^T~~ \dot{\underline{\omega}}^T \right]$. For nonlinear
finite-element analysis, the discretized form of
displacement, velocity, and acceleration are written as
\begin{align}
	\label{DiscretizedDisp}
	\underline{q} (x_1) &= \underline{\underline{N}} ~\hat{\underline{q}}~~~~\underline{q}^T = \left[ \underline{u}^T~~\underline{p}^T \right] \\
	\label{DiscretizedVel}
	\underline{v}(x_1) &= \underline{\underline{N}}~\hat{\underline{v}}~~~~\underline{v}^T = \left[\underline{\dot{u}}^T~~\underline{\omega}^T \right] \\
	\label{DiscretizedAcc}
	\underline{a}(x_1) &= \underline{\underline{N}}~ \hat{\underline{a}}~~~~\underline{a}^T = \left[ \ddot{\underline{u}}^T~~\dot{\underline{\omega}}^T \right]	
\end{align}
where $\tens{N}$ is the shape function matrix and $(\hat{\cdot})$ denotes a
column matrix of nodal values.

To accommodate the FAST modular framework, the governing equations 
\eqref{GovernGEBT-1} and \eqref{GovernGEBT-2} need to be reformulated
into a state-space form. Firstly we recast these equations in compact form
as
\begin{equation}
    \label{CompactGovernGEBT}
    \underline{\mathcal{F}}^I - \underline{\mathcal{F}}^{C\prime} + \underline{\mathcal{F}}^D = \underline{\mathcal{F}}^{ext}
\end{equation}
where $\underline{\mathcal{F}}^I, \underline{\mathcal{F}}^C$ and $\underline{\mathcal{F}}^D$, and $\underline{\mathcal{F}}^{ext}$ are the inertial forces, elastic forces, and externally applied forces, respectively; their definitions are
\begin{align}
    \label{InertialForce}
    \underline{\mathcal{F}}^I &= \begin{Bmatrix}
    \dot{\underline{h}} \\
    \dot{\underline{g}}
    \end{Bmatrix}
    + \begin{bmatrix}
    \underline{\underline{0}} & \underline{\underline{0}} \\
    \dot{\tilde{u}}  &  \underline{\underline{0}}
    \end{bmatrix}
    \begin{Bmatrix}
    \vec{h} \\
    \vec{g}
    \end{Bmatrix} \\
    \label{ElasticForceFc}
     \underline{\mathcal{F}}^C &= \begin{Bmatrix}
    \underline{F} \\
    \underline{M}
    \end{Bmatrix} \\
    \label{ElasticForceFd}
    \underline{\mathcal{F}}^D &= \begin{Bmatrix}
    \underline{0} \\
    (\tilde{x}^\prime_0 + \tilde{u}^\prime)^T \underline{F}
    \end{Bmatrix} \\
    \label{AppliedForce}
    \underline{\mathcal{F}}^{ext} &= \begin{Bmatrix}
    \underline{f} \\
    \underline{m}
    \end{Bmatrix}
\end{align}   
Along with the constitutive equations \eqref{ConstitutiveMass} and \eqref{ConstitutiveStiff}, the inertial force $\underline{\mathcal{F}}^I$ can be written explicitly as
\begin{align}
    \underline{\mathcal{F}}^I &= \begin{Bmatrix}
    m \ddot{\underline{u}} + ( \dot{\tilde{\omega}} + \tilde{\omega} \tilde{\omega} ) m \underline{\eta} \\
    m \tilde{\eta} \ddot{\underline{u}} + \underline{\underline{\varrho}} \dot{\underline{\omega}} + \tilde{\omega}  \underline{\underline{\varrho}} \underline{\omega} 
    \end{Bmatrix} \nonumber \\
    \label{InertialForce2}
    &= \begin{bmatrix}
    m \underline{\underline{I}} & m \tilde{\eta}^T \\
    m \tilde{\eta} & \underline{\underline{\varrho}}
    \end{bmatrix}
    \begin{Bmatrix}
    \ddot{\underline{u}} \\
    \dot{\underline{\omega}}
    \end{Bmatrix} + 
    \begin{bmatrix}
    \underline{\underline{0}} & m \tilde{\omega} \tilde{\eta}^T \\
    \underline{\underline{0}} & \tilde{\omega}  \underline{\underline{\varrho}}
    \end{bmatrix}
    \begin{Bmatrix}
    \dot{\underline{u}} \\
    \underline{\omega}
    \end{Bmatrix} \\
    &  \equiv \underline{\underline{\mathfrak{M}}} \underline{a} + \underline{\underline{\mathcal{G}}} \underline{v}  \nonumber     
\end{align}
where $m$ is the mass density per unit span; $\underline{\eta}$ is the
center of mass location;  $\underline{\underline{\varrho}}$ is the moment of
inertia; $\tens{I}$ is the identity matrix. The definitions of the acceleration vector $\underline{a}$ and velocity vector $\underline{v}$ can be found in Eq.~\eqref{DiscretizedAcc} and \eqref{DiscretizedVel}, respectively. By the newly introduced matrices, the compact form of equations of motion can be rewritten as
\begin{equation}
    \label{CompactForm2}
    \underline{\underline{\mathfrak{M}}}~ \underline{a} + f(\underline{q},\underline{v},t) = 0
\end{equation}
where
\begin{equation}
    \label{CompactForm3}
    f(\underline{q},\underline{v},t) = \underline{\mathcal{F}}^F - \underline{\mathcal{F}}^{C\prime} + \underline{\mathcal{F}}^D - \underline{\mathcal{F}}^{ext}
\end{equation}
\begin{align}
    \underline{\mathcal{F}}^F &= \underline{\underline{\mathcal{G}}} \underline{v}  \nonumber \\
    \label{CompactForm4}
    &= \begin{bmatrix}
    \underline{\underline{0}} & m \tilde{\omega} \tilde{\eta}^T \\
    \underline{\underline{0}} & \tilde{\omega}  \underline{\underline{\varrho}}
    \end{bmatrix}
    \begin{Bmatrix}
    \dot{\underline{u}} \\
    \underline{\omega}
    \end{Bmatrix} 
\end{align}

A weighted residual formulation will be used to enforce the the dynamic equilibrium conditions in Eq.~\eqref{CompactForm2}
\begin{equation}
    \label{FEM-1}
    \int_0^l \tens{N}^T (\tens{\mathfrak{M}} \vec{a} + \vec{\mathcal{F}}^F - \vec{\mathcal{F}}^{C\prime} + \vec{\mathcal{F}}^D - \vec{\mathcal{F}}^{ext}) d x_1=0
\end{equation}
The above equation can be recast as
\begin{equation}
    \label{FEM-2}
    \tens{M} \hat{a} = F(\vec{q},\vec{v},t)
\end{equation}
where
\begin{align}
    \label{FEM-3}
    \tens{M} &= \int_0^l \tens{N}^T \tens{\mathfrak{M}}~ \tens{N}~dx_1 \\
    \label{FEM-4}
    \vec{F}(\vec{q},\vec{v},t) &= \int_0^l \tens{N}^T (-\vec{\mathcal{F}}^F + \vec{\mathcal{F}}^{C\prime} - \vec{\mathcal{F}}^D + \vec{\mathcal{F}}^{ext}) d x_1
\end{align}

To derive the state-space form of the governing equations, \kill{a new
first-order variable} $\vec{x}(t)$ is introduced as
\begin{equation}
    \label{StateSpaceX}
    \vec{x}(t) \equiv \begin{Bmatrix}
    \vec{q}(t) \\
    \vec{v}(t)
    \end{Bmatrix} 
\end{equation}
It is noted that the second component of $\vec{x}(t)$ is not $\vec{\dot{q}}$ but $\vec{v}$ in that the angular velocity $\vec{\omega}$ cannot be calculated as time derivative of the rotation parameter $\vec{p}$. Substituting the discretized quantities in Eqs.~\eqref{DiscretizedDisp} to \eqref{DiscretizedAcc} into Eq.~\eqref{StateSpaceX} and using the relation
\begin{equation}
    \label{AccVel}
    \vec{a} = \vec{\dot{v}} = \begin{Bmatrix}
    \vec{\ddot{u}} \\
    \vec{\dot{\omega}}
    \end{Bmatrix}
\end{equation}
The state-space form can be obtained as
\begin{align}
    \label{StateSpaceGov-1}
    \dot{\hat{\vec{x}}}(t) &= \mathfrak{f}(\hat{\vec{x}}(t),t) \\
    \label{StateSpaceGov-2}
    \hat{\vec{x}}(0) &= \hat{\vec{x}}_0
\end{align}
where
\begin{align}
    \label{StateSpaceGov-3}
    \mathfrak{f}(\hat{\vec{x}}(t),t) &= \tens{A}^{-1} (\hat{\vec{x}}(t)) \vec{b}(\hat{\vec{x}}(t),t) \\
    \label{StateSpaceGov-4}
    \tens{A} (\hat{\vec{x}}(t)) &= \begin{bmatrix}
    \tens{I} & \tens{0} \\
    \tens{0} & \tens{M}
    \end{bmatrix}  \\
    \label{StateSpaceGov-5}
    \vec{b}(\hat{\vec{x}}(t),t) &= \begin{Bmatrix}
    \dot{\hat{\vec{q}}} \\
    \vec{F}(\hat{\vec{x}}(t),t)
    \end{Bmatrix} \\
    \label{StateSpaceGov-6}
    \hat{\vec{x}}_0 &= \begin{Bmatrix}
    \hat{\vec{q}}_0 \\
    \hat{\vec{v}}_0
    \end{Bmatrix}
\end{align}
It is noted that the state-space form, in
Eq.~\eqref{StateSpaceGov-1} and \eqref{StateSpaceGov-2}, can be solved with
any number of first-order ordinary differential equations (ODE) integrators
for first-order-in-time systems. 

 
 \section{Content of Full Paper} In the full paper, we will provide more
details of the theoretical derivations along with various numerical
examples. Different time integration schemes, including explicit and
implicit methods, will be implemented and examined in the numerical studies.
Moreover, we will briefly introduce the format of the current code, which is
in accordance to the FAST modularization framework. \mas{Finally,  BeamDyn
will be validated against experimental data for a realistic wind turbine
blade.}


\bibliographystyle{aiaa}
\bibliography{references}

\end{document}
