\documentclass{aiaa-tc}

\usepackage{color}
\usepackage{amsmath}
%\usepackage{overcite}
\usepackage{graphicx}
\usepackage{subfig}
\usepackage{authblk}
\usepackage{amsfonts}

\input basic.ltx
\def\directory{EPSF/}

%------------------------------------------------------------------------------
% MAS Additions
\newcommand{\mas}[1]{\textcolor{magenta}{#1}}
\newcommand{\qw}[1]{\textcolor{blue}{#1}}
%useful for showing deleted text
\renewcommand{\kill}[1]{\textcolor{red}{\sout{#1}}}                             

\usepackage[normalem]{ulem}  % mas addition
\newcommand{\uvec}[1]{\bar{#1}}
\newcommand{\tens}[1]{\underline{\underline{#1}}}
\renewcommand{\vec}[1]{\underline{#1}}
\renewcommand{\skew}[1]{\widetilde{#1}}
%------------------------------------------------------------------------------

\title{BeamDyn: A High-Fidelity Wind Turbine Blade Solver in the FAST
Modular Framework}


\author[1]{Qi Wang\thanks{Research Engineer, National Wind Technology Center, AIAA Senior Member. Email: Qi.Wang2@nrel.gov}}
\author[2]{Nick Johnson\thanks{Graduate Research Assistant, Department of
Mechanical Engineering.}}
 \author[1]{Michael A. Sprague\thanks{Senior Research Scientist, 
Computational Science Center.}}
 \author[1]{Jason Jonkman\thanks{Senior Engineer, National Wind Technology Center.}}
 \affil[1]{National Renewable Energy Laboratory, Golden, CO 80401}
 \affil[2]{Colorado School of Mines, Golden, CO 80401}
 
 \renewcommand\Authands{ and }

\begin{document}

\maketitle

\begin{abstract}
{BeamDyn, a Legendre-spectral-finite-element implementation of geometrically
exact beam theory (GEBT), is developed to meet the design challenges
associated with highly flexible composite wind turbine blades. In this
paper, the governing equations of GEBT are reformulated into a nonlinear
state-space form to support its coupling within the
modular framework of the FAST wind turbine
computer-aided-engineering (CAE) tool. Different time integration
schemes (implicit and explicit) are implemented and examined for wind
turbine analysis.  Numerical examples are presented to show the capability of this new beam solver. A validation example of a realistic wind turbine blade CX-100 is also presented. Some practical procedures on modeling wind turbines are discussed.}     
\end{abstract}

\section{Introduction} Wind power installations in the U.S. have exceeded 60
GW, and have become an increasingly important part of the overall energy
portfolio. In recent years, the size of wind turbines has also increased in
the quest for economies of scale.  Larger wind turbine blades result in
structures that are highly flexible.  To ensure the performance and
reliability of wind turbines, it is crucial to make use of
computer-aided-engineering (CAE) tools that are capable of analyzing wind
turbine blades in an accurate and efficient manner. Although modern
computers enable three-dimensional (3D) analysis
of a fully resolved blade, such analyses are too
expensive for iterative design. More importantly, modern composite wind
turbine blades are very well suited to nonlinear beam models, which can
capture with high-fidelity the deformation response under realistic
operating conditions, and in a small fraction of the time required by a
fully resolved 3D simulation. 

Beam models are widely used to analyze structures that have one of its
dimensions being much larger than the other two.  Many engineering
structures are modeled as beams, e.g.,  bridges, joists, and helicopter
rotor blades.  Similarly, beam models are well suited to analyze, with high
fidelity,  wind turbine blades, towers, and shafts.  Most wind turbine
blades are constructed of composite materials, and analysis of composite
beams is more complicated than isotropic beams due to the elastic coupling
effects.  The geometrically exact beam theory (GEBT), first proposed by
Reissner\cite{Ressiner1973}, is a beam-deformation model capable of
enabling efficient analysis of highly flexible composite structures.
GEBT has demonstrated its efficacy in helicopter rotor analysis.
Simo\cite{Simo1985} and Simo and Vu-Quoc\cite{Simo1986} extended Reissner's
initial work to include 3D dynamic problems. Jeleni\'c and
Crisfield\cite{Crisfield1999} derived a finite-element (FE) method that
interpolates the rotation field thereby preserving the geometric exactness
of this theory. It is noted that Ibrahimbegovi\'c and his colleagues
implemented this theory for static\cite{Ibrahim1995} and
dynamic\cite{Ibrahim1998} analysis. Readers are referred to
Hodges\cite{HodgesBeamBook}, where comprehensive derivations and discussions
on nonlinear composite-beam theories can be found. Recently, a mixed
formulation of GEBT along with the numerical implementation was presented by
Yu and Blair\cite{YuGEBT}.  

FAST is a CAE tool developed by the National Renewable Energy Laboratory
(NREL) for the purposes of wind turbine analysis for both land-based and
offshore wind turbines using realistic operating conditions.  The current
beam model in FAST is not capable of predictive analysis of highly flexible
composite wind turbine blades.
Recently, FAST has been reformulated under a
new modularized framework that provides a rigorous means by which various
mathematical systems are implemented in distinct modules and are interconnected
to solve for the global, coupled, dynamic response of wind turbines and wind
plants \cite{Jonkman:2013,website:FASTModularizationFramework}.

In this paper, a three-dimensional displacement-based
implementation of the geometrically exact beam theory using Legendre
spectral finite elements is presented. The theory is reformulated in a
nonlinear state-space form for the purpose of integrating with the FAST
framework, thereby introducing an optional high-fidelity beam model as an
alternative to the current beam model. This work builds on previous efforts
that showed the implementation GEBT and spatial discretization executed
using Legendre spectral finite elements
(LSFEs)\cite{Wang:GEBT2013,Wang:SFE2013,Wang:GEBT2014,Sprague:FAST2014} for
analysis of composite wind turbine blades. The paper is organized as
follows.  First, the theoretical foundation of the geometrically exact beam
theory along with the reformulation of the governing equations into a
state-space form is introduced. Coupling to the FAST framework is then
discussed. Finally, numerical examples are
provided to verify and validate the accuracy and efficiency of the present model for
composite wind turbine blades. 

~

\section{Geometrically Exact Beam Theory}

This section reviews the geometric exact beam theory for completeness of this paper. The content of this section can be found in many other papers and textbooks.
Figure~\ref{Kinematics} shows a beam in its initial undeformed
and deformed states. A reference frame $\mathbf{b}_i$ is introduced along the
beam axis for the undeformed state; a frame $\mathbf{B}_i$ is introduced
along each point of the deformed beam axis. Curvilinear coordinate $x_1$ defines the intrinsic parameterization of the reference line.
\begin{figure}
\centering
\includegraphics[width=5.0in]{\directory Kinematics.eps}
\caption{Schematic of beam deformation} \label{Kinematics}
\end{figure}
In this paper, we use matrix notation to denote vectorial or vectorial-like quantities. For example, we use a underline to denote a vector $\underline{u}$, a bar to denote unit vector $\bar{n}$, and double underline to denote a tensor $\underline{\underline{\Delta}}$. Note that sometimes the underlines only denote the dimension of the corresponding matrix. The governing equations of motion for geometric exact beam theory can be written as \cite{Bauchau:2010}
\begin{align}
	\label{GovernGEBT-1}
	\dot{\underline{h}} - \underline{F}^\prime &= \underline{f} \\
	\label{GovernGEBT-2}
	\dot{\underline{g}} + \dot{\tilde{u}} \underline{h} - \underline{M}^\prime + (\tilde{x}_0^\prime + \tilde{u}^\prime)^T \underline{F} &= \underline{m}
\end{align}
where $\vec{h}$ and $\vec{g}$ are the linear and angular momenta resolved in the inertial coordinate system, respectively; $\vec{F}$ and $\vec{M}$ are the beam's sectional forces and moments, respectively; $\vec{u}$ is the 1D displacement of the reference line; $\vec{x}_0$ is the position vector of a point along the beam's reference line; $\vec{f}$ and $\vec{m}$ are the distributed force and moment applied to the beam structure.  Notation $(\bullet)^\prime$ indicates a derivative with respect to the beam axis $x_1$ and $\dot{(\bullet)}$ indicates a derivative with respect to time. The tilde operator $(\skew{\bullet})$ defines a second-order, skew-symmetric tensor corresponding to the given vector. In the literature, it is also termed as "cross-product matrix". For example,
\[
	\skew{n} = 
	     		\begin{bmatrix}
			0 & -n_3 & n_2 \\
			n_3 & 0 & -n_1 \\
			-n_2 & n_1 & 0\\
			\end{bmatrix}	
\]
The constitutive equations relate the velocities to the momenta and the one-dimensional strain measures to the sectional resultants as
\begin{align}
	\label{ConstitutiveMass}
	\begin{Bmatrix}
	\underline{h} \\
	\underline{g}
	\end{Bmatrix}
	= \underline{\underline{\mathcal{M}}} \begin{Bmatrix}
	\dot{\underline{u}} \\
	\underline{\omega}
	\end{Bmatrix} \\
	\label{ConstitutiveStiff}
	\begin{Bmatrix}
	\underline{F} \\
	\underline{M}
	\end{Bmatrix}
	= \underline{\underline{\mathcal{C}}} \begin{Bmatrix}
	\underline{\epsilon} \\
	\underline{\kappa}
	\end{Bmatrix}
\end{align}
where $\underline{\underline{\mathcal{M}}}$ and
$\underline{\underline{\mathcal{C}}}$ are the $6 \times 6$ sectional mass
and stiffness matrices, respectively (note that they are not really tensors);
$\underline{\epsilon}$ and $\underline{\kappa}$ are the 1D strains and
curvatures, respectively. $\underline{\omega}$ is the angular velocity
vector that is defined by the rotation tensor $\underline{\underline{R}}$ as
$\underline{\omega} =
axial(\dot{\underline{\underline{R}}}~\underline{\underline{R}}^T)$. The 1D
strain measures are defined as
\begin{equation}
    \label{1DStrain}
    \begin{Bmatrix}
        \vec{\epsilon} \\
        \vec{\kappa}
    \end{Bmatrix}
    =
    \begin{Bmatrix}
        \vec{x}^\prime_0 + \vec{u}^\prime - (\tens{R} ~\tens{R}_0) \bar{\imath}_1 \\
        \vec{k}
    \end{Bmatrix}
\end{equation}
where $\vec{k} = \text{axial} [(\tens{R R_0})^\prime (\tens{R R_0})^T]$ is the sectional
curvature vector resolved in the inertial basis and $\bar{\imath}_1$ is the unit
vector along $x_1$ direction in the inertial basis. It is noted that the
three sets of equations, including equations of motion
Eq.~\eqref{GovernGEBT-1} and \eqref{GovernGEBT-2}, constitutive equations
Eq.~\eqref{ConstitutiveMass} and \eqref{ConstitutiveStiff}, and kinematical
equations Eq.~\eqref{1DStrain}, provided a full mathematical description of elasticity problems. 

For a displacement-based finite element implementation, there are six
degree-of-freedoms at each node: three displacement components and three
rotation components. Here, we use $\vec{q}$ to denote the elemental
displacement array as $\underline{q}^T=\left[
\underline{u}^T~~\underline{p}^T\right]$ where $\vec{u}$ is the
displacement and $\vec{p}$ is the rotation-parameter vector. The
acceleration array can thus be defined as $\underline{a}^T=\left[
\ddot{\underline{u}}^T~~ \dot{\underline{\omega}}^T \right]$. For nonlinear
finite-element analysis, the discretized form of
displacement, velocity, and acceleration are written as
\begin{align}
	\label{DiscretizedDisp}
	\underline{q} (x_1) &= \underline{\underline{N}} ~\hat{\underline{q}}~~~~\underline{q}^T = \left[ \underline{u}^T~~\underline{p}^T \right] \\
	\label{DiscretizedVel}
	\underline{v}(x_1) &= \underline{\underline{N}}~\hat{\underline{v}}~~~~\underline{v}^T = \left[\underline{\dot{u}}^T~~\underline{\omega}^T \right] \\
	\label{DiscretizedAcc}
	\underline{a}(x_1) &= \underline{\underline{N}}~ \hat{\underline{a}}~~~~\underline{a}^T = \left[ \ddot{\underline{u}}^T~~\dot{\underline{\omega}}^T \right]	
\end{align}
where $\tens{N}$ is the shape function matrix and $(\hat{\cdot})$ denotes a
column matrix of nodal values.

To accommodate the FAST modular framework, the governing equations 
\eqref{GovernGEBT-1} and \eqref{GovernGEBT-2} need to be reformulated
into a state-space form. Firstly we recast these equations in compact form
as
\begin{equation}
    \label{CompactGovernGEBT}
    \underline{\mathcal{F}}^I - \underline{\mathcal{F}}^{C\prime} + \underline{\mathcal{F}}^D = \underline{\mathcal{F}}^{ext}
\end{equation}
where $\underline{\mathcal{F}}^I, \underline{\mathcal{F}}^C$ and $\underline{\mathcal{F}}^D$, and $\underline{\mathcal{F}}^{ext}$ are the inertial forces, elastic forces, and externally applied forces, respectively; their definitions are
\begin{align}
    \label{InertialForce}
    \underline{\mathcal{F}}^I &= \begin{Bmatrix}
    \dot{\underline{h}} \\
    \dot{\underline{g}}
    \end{Bmatrix}
    + \begin{bmatrix}
    \underline{\underline{0}} & \underline{\underline{0}} \\
    \dot{\tilde{u}}  &  \underline{\underline{0}}
    \end{bmatrix}
    \begin{Bmatrix}
    \vec{h} \\
    \vec{g}
    \end{Bmatrix} \\
    \label{ElasticForceFc}
     \underline{\mathcal{F}}^C &= \begin{Bmatrix}
    \underline{F} \\
    \underline{M}
    \end{Bmatrix} \\
    \label{ElasticForceFd}
    \underline{\mathcal{F}}^D &= \begin{Bmatrix}
    \underline{0} \\
    (\tilde{x}^\prime_0 + \tilde{u}^\prime)^T \underline{F}
    \end{Bmatrix} \\
    \label{AppliedForce}
    \underline{\mathcal{F}}^{ext} &= \begin{Bmatrix}
    \underline{f} \\
    \underline{m}
    \end{Bmatrix}
\end{align}   
Along with the constitutive equations \eqref{ConstitutiveMass} and \eqref{ConstitutiveStiff}, the inertial force $\underline{\mathcal{F}}^I$ can be written explicitly as
\begin{align}
    \underline{\mathcal{F}}^I &= \begin{Bmatrix}
    m \ddot{\underline{u}} + ( \dot{\tilde{\omega}} + \tilde{\omega} \tilde{\omega} ) m \underline{\eta} \\
    m \tilde{\eta} \ddot{\underline{u}} + \underline{\underline{\varrho}} \dot{\underline{\omega}} + \tilde{\omega}  \underline{\underline{\varrho}} \underline{\omega} 
    \end{Bmatrix} \nonumber \\
    \label{InertialForce2}
    &= \begin{bmatrix}
    m \underline{\underline{I}} & m \tilde{\eta}^T \\
    m \tilde{\eta} & \underline{\underline{\varrho}}
    \end{bmatrix}
    \begin{Bmatrix}
    \ddot{\underline{u}} \\
    \dot{\underline{\omega}}
    \end{Bmatrix} + 
    \begin{bmatrix}
    \underline{\underline{0}} & m \tilde{\omega} \tilde{\eta}^T \\
    \underline{\underline{0}} & \tilde{\omega}  \underline{\underline{\varrho}}
    \end{bmatrix}
    \begin{Bmatrix}
    \dot{\underline{u}} \\
    \underline{\omega}
    \end{Bmatrix} \\
    &  \equiv \underline{\underline{\mathfrak{M}}} \underline{a} + \underline{\underline{\mathcal{G}}} \underline{v}  \nonumber     
\end{align}
where $m$ is the mass density per unit span; $\underline{\eta}$ is the
center of mass location;  $\underline{\underline{\varrho}}$ is the moment of
inertia; $\tens{I}$ is the identity matrix. The definitions of the acceleration vector $\underline{a}$ and velocity vector $\underline{v}$ can be found in Eq.~\eqref{DiscretizedAcc} and \eqref{DiscretizedVel}, respectively. By the newly introduced matrices, the compact form of equations of motion can be rewritten as
\begin{equation}
    \label{CompactForm2}
    \underline{\underline{\mathfrak{M}}}~ \underline{a} + f(\underline{q},\underline{v},t) = 0
\end{equation}
where
\begin{equation}
    \label{CompactForm3}
    f(\underline{q},\underline{v},t) = \underline{\mathcal{F}}^F - \underline{\mathcal{F}}^{C\prime} + \underline{\mathcal{F}}^D - \underline{\mathcal{F}}^{ext}
\end{equation}
\begin{align}
    \underline{\mathcal{F}}^F &= \underline{\underline{\mathcal{G}}} \underline{v}  \nonumber \\
    \label{CompactForm4}
    &= \begin{bmatrix}
    \underline{\underline{0}} & m \tilde{\omega} \tilde{\eta}^T \\
    \underline{\underline{0}} & \tilde{\omega}  \underline{\underline{\varrho}}
    \end{bmatrix}
    \begin{Bmatrix}
    \dot{\underline{u}} \\
    \underline{\omega}
    \end{Bmatrix} 
\end{align}

A weighted residual formulation will be used to enforce the the dynamic equilibrium conditions in Eq.~\eqref{CompactForm2}
\begin{equation}
    \label{FEM-1}
    \int_0^l \tens{N}^T (\tens{\mathfrak{M}} \vec{a} + \vec{\mathcal{F}}^F - \vec{\mathcal{F}}^{C\prime} + \vec{\mathcal{F}}^D - \vec{\mathcal{F}}^{ext}) d x_1=0
\end{equation}
The above equation can be recast as
\begin{equation}
    \label{FEM-2}
    \tens{M} \hat{a} = F(\vec{q},\vec{v},t)
\end{equation}
where
\begin{align}
    \label{FEM-3}
    \tens{M} &= \int_0^l \tens{N}^T \tens{\mathfrak{M}}~ \tens{N}~dx_1 \\
    \label{FEM-4}
    \vec{F}(\vec{q},\vec{v},t) &= \int_0^l \tens{N}^T (-\vec{\mathcal{F}}^F + \vec{\mathcal{F}}^{C\prime} - \vec{\mathcal{F}}^D + \vec{\mathcal{F}}^{ext}) d x_1
\end{align}

To derive the state-space form of the governing equations,  $\vec{x}(t)$ is introduced as
\begin{equation}
    \label{StateSpaceX}
    \vec{x}(t) \equiv \begin{Bmatrix}
    \vec{q}(t) \\
    \vec{v}(t)
    \end{Bmatrix} 
\end{equation}
It is noted that the second component of $\vec{x}(t)$ is not $\vec{\dot{q}}$ but $\vec{v}$ in that the angular velocity $\vec{\omega}$ cannot be calculated as time derivative of the rotation parameter $\vec{p}$. Substituting the discretized quantities in Eqs.~\eqref{DiscretizedDisp} to \eqref{DiscretizedAcc} into Eq.~\eqref{StateSpaceX} and using the relation
\begin{equation}
    \label{AccVel}
    \vec{a} = \vec{\dot{v}} = \begin{Bmatrix}
    \vec{\ddot{u}} \\
    \vec{\dot{\omega}}
    \end{Bmatrix}
\end{equation}
The state-space form can be obtained as
\begin{align}
    \label{StateSpaceGov-1}
    \dot{\hat{\vec{x}}}(t) &= \mathfrak{f}(\hat{\vec{x}}(t),t) \\
    \label{StateSpaceGov-2}
    \hat{\vec{x}}(0) &= \hat{\vec{x}}_0
\end{align}
where
\begin{align}
    \label{StateSpaceGov-3}
    \mathfrak{f}(\hat{\vec{x}}(t),t) &= \tens{A}^{-1} (\hat{\vec{x}}(t)) \vec{b}(\hat{\vec{x}}(t),t) \\
    \label{StateSpaceGov-4}
    \tens{A} (\hat{\vec{x}}(t)) &= \begin{bmatrix}
    \tens{I} & \tens{0} \\
    \tens{0} & \tens{M}
    \end{bmatrix}  \\
    \label{StateSpaceGov-5}
    \vec{b}(\hat{\vec{x}}(t),t) &= \begin{Bmatrix}
    \dot{\hat{\vec{q}}} \\
    \vec{F}(\hat{\vec{x}}(t),t)
    \end{Bmatrix} \\
    \label{StateSpaceGov-6}
    \hat{\vec{x}}_0 &= \begin{Bmatrix}
    \hat{\vec{q}}_0 \\
    \hat{\vec{v}}_0
    \end{Bmatrix}
\end{align}
It is noted that the state-space form, in
Eq.~\eqref{StateSpaceGov-1} and \eqref{StateSpaceGov-2}, can be solved with
any number of first-order ordinary differential equations (ODE) integrators
for first-order-in-time systems. 

\section{Numerical Examples}

\subsection{Initially twisted/curved beams}

A initial twisted beam is examined first. A straight beam ($k_2=k_3=0$) with an initial twist ($k_1\neq 0$) is shown in Figure~\ref{Twist1}. The beam is linearly twisted from 0 degree twist at the root to 90 degree at the tip, and the twist is in the positive $\theta_1$ direction.
\begin{figure}
\centering
\includegraphics[width=5.0in]{\directory Twist1.eps}
\caption{Sketch of a initially twisted beam} 
\label{Twist1}
\end{figure}
 Table~\ref{E1Property} shows the material properties for A36 steel, the geometry, and force applied to the beam. The height and base values reported in the table are the height and base of the rectangular cross-section. 
 \begin{table}
\caption{\label{E1Property} Properties for twisted beam}
\begin{center}
    \begin{tabular}{| l | l |}
    	\hline
    	Property               & Value   \\ \hline
    	Elastic Modulus                      & 200 GPa \\ \hline
    	Shear Modulus                      & 79.3 GPa \\ \hline
    	Height                      & 0.5 m   \\ \hline
    	Base                      & 0.25 m  \\ \hline
    	Length                      & 10 m    \\ \hline
    	Force                      & 4000 kN \\ \hline
    \end{tabular}
\end{center}
\end{table}
The beam discretized using one 20\textsuperscript{th}-order LSFE. The results for the twisted beam are shown in Table~\ref{E1u} and compared to the baseline results obtained from a solid ANSYS model. It can be seen that the error between the BeamDyn simulation and the ANSYS baseline solution are very close. 
\begin{table}
\caption{\label{E1u} Comparison of tip displacements of a initially twisted beam} 
\begin{center} 
    \begin{tabular}{| l | l | l | l | l | l | l |}
    	\hline
    	        & $u_1$ (m) & $u_2$ (m) & $u_3$ (m)  \\ \hline
    	BeamDyn  & -1.132727     & -1.715123       & -3.578671      \\  \hline
    	ANSYS   & -1.134192     & -1.714467      & -3.584232     \\ \hline
    	Percent Error   & 0.129     & 0.038      & 0.155     \\ \hline
    \end{tabular}
\end{center}
\end{table} 

Next a beam with initial curvature is examined. It is clear that the initial curvature plays a major role in the distribution of the elastic forces within the beam. As such it is very important to ensure that BeamDyn is capable of modeling this effect properly. A benchmark problem for a curved beam is the case proposed by Bathe \cite{Bathe1979}, and is used here as a verification case. Figure~\ref{E1Curved} shows the configuration of the cantilevered curved beam. The beam is in the $x_1$, $x_2$ plane, and in the positive $x_1$ direction and negative $x_2$ direction. A force of 600 lbs is applied in the positive $x_3$ direction. The beam is defined by the 45 degree arc of a 100 inch radius centered at 100 inches in the negative $x_2$ direction.  The geometry of the cross section for the curved beam is square, and the material properties can be found in \cite{Bathe1979}. 
\begin{figure}
\centering
\includegraphics[width=3.0in]{\directory E1Curved.eps}
\caption{Sketch of a initially curveded beam} 
\label{E1Curved}
\end{figure}
The beam is discretized by one 20\textsuperscript{th}-order LSFE. The results for this static analysis are shown in Table~\ref{E1CurvedDisp} and compared to the results published in Bathe \cite{Bathe1979}.

\begin{table}
\caption{\label{E1CurvedDisp} Comparison of tip displacements of a initially curved beam } 
\begin{center}
    \begin{tabular}{| l | l | l | l | l | l | l |}
    	\hline
    	        & $u_1$ (inch) & $u_2$ (inch) & $u_3$ (inch)  \\ \hline
    	BeamDyn (one-LSFE) & -23.6     & 13.4       & 53.3      \\  \hline
    	Published   & -23.5     & 13.4       & 53.4     \\ \hline
    \end{tabular}
\end{center}
\end{table} 

It can be seen from these results that the simulations from BeamDyn for a initially curved beam match quite well with the published results. Through these two numerical cases, it can therefore be said that BeamDyn is capable of modeling beams with initial twist and curvatures.

\subsection{Static analysis of CX-100 blade}
The main utility of BeamDyn will be to analyze anisotropic wind turbine blades, therefore the CX-100 will be analyzed and serve as a validation case. The CX-100 was chosen because it is a well characterized blade with a wealth of publicly available data regarding the construction and material properties of the blade. The CX-100 is a 9 m blade designed by Sandia National Laboratory \cite{paquette2006modeling}.

The VABS cross-sectional properties for this beam were provided by Dr.\ D.J.\ Luscher of Los Alamos National Laboratory. Dr.\ Luscher conducted a similar study with a finite element code based on GEBT theory, called NLBeam \cite{Luscher:2013}. The cross-sectional properties were provided at 40 points along the beam. A typical stiffness matrix is shown at 2.2 m along the span of the blade, and is given by

\begin{align*}
C =10^3 \times \begin{bmatrix}
	193,000 & -75.4   & 12.2   & -75.2  & -1970    & -3500    \\
	-75.4  & 19,500 & 4,760   & 62.6  & 67.3    & 11.3    \\
	12.2  & 4,760   & 7,210 & -450  & 17.0    & 2.68    \\
	-75.2  & 62.6   & -450   & 518 & 1.66    & -1.11    \\
	-1,970  & 67.3   & 17.0   & 1.66  & 2,280 & -879    \\
	-3,500  & 11.6   & 2.68   & -1.11  & -875    & 4,240
\end{bmatrix}
\end{align*}


Figure~\ref{CX100Sketch} \cite{paquette2006modeling} shows the different material lay-ups for the CX-100 blade. Each color represents a section with unique material properties. This figure also shows the geometry of the blade. Figure~\ref{CX100Setup} \cite{paquette2006modeling} shows the test configuration for the static test performed at the National Wind Technology Center (NWTC) in Boulder, Colorado. The whiffle-tree configuration applies the load at 3.00 m, 5.81 m, and 7.26 m from the root of the blade to achieve a maximum root moment of 128.6 kN m. The loads and positions are given in Table~\ref{CX100Load} below.
\begin{figure}
\centering
\includegraphics[width=3.0in]{\directory CX100Sketch.eps}
\caption{Material layup and geometry of CX-100 wind turbine blade} 
\label{CX100Sketch}
\end{figure}

\begin{figure}
\centering
\includegraphics[width=5.0in]{\directory CX100Setup.eps}
\caption{Test configuration for static pull test conducted at the NWTC \cite{paquette2006modeling}} 
\label{CX100Setup}
\end{figure}

\begin{table} 
\caption{\label{CX100Load}Positions and applied loads in CX-100 static loads test at NWTC  } 
\begin{center}
    \begin{tabular}{| l | l |l |}
    	\hline
    	 Saddle \# &     Position (m) & Applied load(kN)  \\ \hline
    1&	3.00 & 16.9         \\  \hline
    2&	5.81   & 5.47         \\ \hline
    3&	    	7.26   & 5.59         \\ \hline
    \end{tabular}
\end{center}
\end{table}

The displacements, $u_3$, at each of the load points were tracked for the experiment and are given in Table~\ref{CX100Results}. The BeamDyn simulation was completed using four 7\textsuperscript{th}-order LSFEs and the results are also given in Table~\ref{CX100Results}.

\begin{table}
\caption{\label{CX100Results}Experimental and BeamDyn simulation results for CX-100 static test  } 
\begin{center}
    \begin{tabular}{| l | l | l | l |}
    	\hline
    	             & $u_3$ at saddle \#1 (m) & $u_3$ at saddle \#2 (m) & $u_3$ at saddle \#3 (m) \\ \hline
    	Experimental & 0.083530             & 0.381996               & 0.632460             \\ \hline
    	BeamDyn      & 0.072056               & 0.381074                & 0.698850           \\ \hline
    	    	Percent error      &        13.74        & 0.24                & 10.5           \\ \hline
    \end{tabular}
\end{center}
\end{table} 
The displacements are plotted in Figure~\ref{CX100Disp}. The maximum tip displacement for the experimental data is 1.03 m, while the maximum tip displacement for the BeamDyn simulation in 1.12 m. This discrepancy is explained in \cite{Luscher:2013} as a difference in the rigidity of the boundary condition when calculating the 2-D sectional properties with VABS. It stands to reason that since we are using the same sectional properties the same errors would be evident, and we experience the same overall effect as the results published in \cite{Luscher:2013}. It should also be noted that the tip displacement was not measured in the experiment but was extrapolated based on the recoded data at 3.00 m, 5.81 m, and 7.26 m. Overall the results are in good agreement.

\begin{figure}
\centering
\includegraphics[width=3.0in]{\directory CX100Disp.eps}
\caption{Displacement $u_3$ along the length of the blade for experimental data and BeamDyn simulation} 
\label{CX100Disp}
\end{figure}

Next a convergence study of the tip displacements is completed for the CX-100 blade in BeamDyn. Figure~\ref{CX100conv3} shows the error as a function of the number of nodes. The percentage error is calculated against the experimental data in Table~\ref{CX100Results}. It can be seen that the convergence rate is not exponential as desired. This is likely a function of the sharp gradients in the sectional constants. Next we mesh this blade so the element boundaries coincide with the locations where the sectional properties are defined. It was stated before that the cross-sectional properties for the CX-100 blade are given at 40 locations along the length of the blade. In order to have an element coincide with each sectional property, we must use thirty-nine LSFEs. Just as we have for the previous simulation, with two-LSFEs, the error in this simulation is found by assigning the benchmark solution as a highly refined solution in BeamDyn. Figure~\ref{CX100elem2} shows the results of this simulation. Each circle on the plot indicates an additional order of the LSFE, i.e.\ the maximum LSFE order is six. It can be seen that we have achieved spectral convergence with this simulation, albeit with many elements. It can therefore be stated that for composite beams with sharp gradients in the cross-sectional stiffness matrix the spectral convergence is compromised. It should be noted here that while the spectral convergence suffers as a result of sharp gradients in the cross-sectional stiffness matrix the simulations still return reasonable results, so the utility of BeamDyn is not compromised in this sense.
  

\begin{figure}
\centering
\includegraphics[width=3.0in]{\directory CX100conv3.eps}
\caption{Error in $u_3$ as a function of the number of nodes} 
\label{CX100conv3}
\end{figure}

\begin{figure}
\centering
\includegraphics[width=3.0in]{\directory CX100elem2.eps}
\caption{Error in $u_3$ compared to a highly refined solution in BeamDyn as a function of the number of nodes for thirty-nine 1\textsuperscript{st} to 6\textsuperscript{th}-order LSFEs coincident with sectional properties} 
\label{CX100elem2}
\end{figure}

\subsection{Dynamics of CX-100 blade}
The final test case is to illustrate that BeamDyn is capable of accurately analyzing dynamic movement. Here the CX-100 blade is given a constant rotational velocity and a gravity force load is applied. A boundary condition is specified where the blade is allowed to rotate about the node located at its root. This test case is analyzed in both BeamDyn and Dymore. The beam is discretized by one 8\textsuperscript{th}-order element in BeamDyn, and forty 3\textsuperscript{rd}-order elements in Dymore. The angular velocity of the blade is $\frac{\pi}{3}$ rad/s. The time integrator for the dynamic case is a Runge-Kutta fourth-order method, and the time step size is $5 \times 10^{-5}$ s. The time integrator for Dymore is the generalized-alpha time integrator, with a time step size of $1 \times 10^{-3}$ s. The total simulation time in both BeamDyn and Dymore is 6 s.

Figure~\ref{E3U} shows the time history for all displacements and rotations given by the BeamDyn and Dymore. For the most part the displacements are in good agreement, and the root mean square error for $u_3$ is 0.0335 which is given by
\begin{equation}
\varepsilon_{RMS}=\sqrt{\frac{\sum_{k=0}^{n_{max}}[u_3^k-u_b(t^k)]^2}{\sum_{k=0}^{n_{max}}[u_b(t^k)]^2}}
\end{equation} 
where, $u_b(t)$ is the benchmark solution and is given by the Dymore solution.
\begin{figure}
    \centering
    \begin{tabular}{c}
    \subfloat[$u_1$]{\label{E3U:u1}\includegraphics[width=3.0 in]{\directory  E3Tipu1.eps}} \qquad
\subfloat[$u_2$]{\label{E3U:u2}\includegraphics[width=3.0in]{\directory  E3Tipu2.eps}} \\
\subfloat[$u_3$]{\label{E3U:u3}\includegraphics[width=3.0 in]{\directory  E3Tipu3.eps}} \qquad
\subfloat[$p_1$]{\label{E3U:p1}\includegraphics[width=3.0 in]{\directory  E3Tipp1.eps}} \\
\subfloat[$p_2$]{\label{E3U:p2}\includegraphics[width=3.0 in]{\directory  E3Tipp2.eps}} \qquad
\subfloat[$p_3$]{\label{E3U:p3}\includegraphics[width=3.0 in]{\directory  E3Tipp3.eps}} \\
\end{tabular}
\caption{Tip displacement and rotation histories of a CX-100 blade rotating at a constant speen}
\label{E3U}
\end{figure} 
Figure~\ref{E3RootM2} shows the root force for the no gravity load applied and gravity load applied cases. It can be seen that the root forces are higher for the case where the gravity force is applied, as expected.
\begin{figure}
\centering
\includegraphics[width=3.0in]{\directory E3RootM2.eps}
\caption{Root force for dynamic simulation of BeamDyn with gravity load and without gravity load} 
\label{E3RootM2}
\end{figure}

 \section{Conclusion}
  In this paper, we reported the latest development of BeamDyn, a beam solver in FAST modularization framework. The geometrically exact beam theory has been reformulated into the first-order state-space form so first-order time integrators, as required by FAST framework, can be applied to discretize the problem in the time domain. Numerical examples are provided to validate the capability of BeamDyn against initially curved/twisted beams. A validation example is also presented where the numerical results are compared with experimental data. Good agreement can be observed. Finally, we studied the dynamic behavior of this realistic CX-100 wind turbine blade by prescribing a rotation at its root. Different time integrators are examined.
  
\section*{Acknowledgments} 

This work was supported by the U.S. Department of Energy under Contract No.\
DE-AC36-08-GO28308 with the National Renewable Energy Laboratory. Support
was partially provided through a Laboratory Directed Research and Development grant
\textit{High-Fidelity Computational Modeling of Wind-Turbine Structural
Dynamics}.
  
  

\bibliographystyle{aiaa}
\bibliography{references}

\end{document}
