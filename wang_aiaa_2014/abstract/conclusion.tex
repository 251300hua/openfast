%&LaTeX
\section{Conclusion}

This paper presents a displacement-based implementation of geometrically
exact beam theory for three-dimensional nonlinear elastic deformation. Legendre spectral finite elements are adopted for spatial discretization of
the beam. Numerical examples were presented that demonstrate the capability
of BeamDyn, a LSFE beam solver for wind turbine analysis developed at NREL. A benchmark static problem for nonlinear deformation of a beam was studied first. The agreement between the results calculated by BeamDyn and the analytical solution are excellent. Moreover, a convergence study was conducted, where the convergence rate of Legendre spectral elements were compared with conventional quadratic finite elements. Exponential convergence rates were observed as expected for this type of element. A composite cantilever beam was studied both statically and dynamically. The static results are verified against those obtained by Dymore. The elastic coupling effects were shown in these two cases. It concludes that BeamDyn is a powerful tool for composite beam analysis that can be used as a wind turbine blade module in the FAST modularization framework. Future work will involve more verification and validation of BeamDyn and coupling of BeamDyn to FAST, with completion expected in June 2014. 

