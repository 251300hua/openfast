\documentclass{aiaa-tc}

\usepackage{color}
\usepackage{amsmath}
%\usepackage{overcite}
\usepackage{graphicx}
\usepackage{subfig}
\usepackage{authblk}
\usepackage{psfrag}
%\usepackage{leftidx}
\usepackage{fouridx}
\usepackage{url}


\input basic.ltx
\def\directory{EPSF/}

%------------------------------------------------------------------------------
% MAS Additions
\newcommand{\mas}[1]{\textcolor{magenta}{COMMENT: #1}}
\newcommand{\qw}[1]{\textcolor{blue}{#1}}
%useful for showing deleted text
\renewcommand{\kill}[1]{\textcolor{red}{\sout{#1}}}                             

\usepackage[normalem]{ulem}  % mas addition
\newcommand{\uvec}[1]{\bar{#1}}
\newcommand{\tens}[1]{\underline{\underline{#1}}}
\renewcommand{\vec}[1]{\underline{#1}}
\renewcommand{\skew}[1]{\widetilde{#1}}
%------------------------------------------------------------------------------
\newcommand\Prefix[3]{\vphantom{#3}#1#2#3}

%\title{A Efficient High-Fidelity Beam Solver in FAST Modularization Framework}

\title{Nonlinear Legendre Spectral Finite Elements for Wind Turbine Blade
Dynamics\thanks{The submitted manuscript has been offered by employees of
the
Alliance for Sustainable Energy, LLC (Alliance), a contractor of the U.S.\
Government under Contract No.\ DE-AC36-08GO28308.  Accordingly, the U.S.\
Government and Alliance retain a nonexclusive royalty-free license to
publish or reproduce the published form of this contribution, or allow
others to do so, for U.S.\ Government purposes.}}
%{Rotation parameterization for analysis of geometrically nonlinear deformation of beams}

%{Structural Dynamics Analysis of Geometric Nonlinear Composite Beams}

% typeset with pdflatex, bibtex, pdflatex, pdflatex


%\author{Qi Wang%
 %        \thanks{Postdoctoral Research Fellow, AIAA Member.}\\
 %        \normalsize\itshape
 %        National Renewable Energy Laboratory, Golden, CO 80020}



 \author[1]{Qi Wang\thanks{Research Engineer, National Wind Technology Center, AIAA Member. Email: Qi.Wang2@nrel.gov}}
 \author[1]{Michael A. Sprague\thanks{Senior Research Scientist, Scientific
Computing Center, AIAA Member. Email: Michael.A.Sprague@nrel.gov}}
 \author[1]{Jason Jonkman\thanks{Senior Engineer, National Wind Technology
Center, AIAA Member. Email: Jason.Jonkman@nrel.gov.}}
 \author[2]{Nick Johnson\thanks{Graduate Research Assistant, Department of
Mechanical Engineering.}}
 \affil[1]{National Renewable Energy Laboratory, Golden, CO 80401}
 \affil[2]{Colorado School of Mines, Golden, CO 80401}
 
 \renewcommand\Authands{ and }

\sloppy
\begin{document}

\maketitle

\begin{abstract}
This paper presents a numerical implementation and evaluation of a new
nonlinear beam finite element model appropriate for highly flexible
wind turbine blades made of composite materials.  The underlying model uses
the geometrically exact beam theory (GEBT) and spatial discretization is
accomplished with Legendre spectral finite elements (LSFEs).  The
displacement-based GEBT is presented, which includes the coupling effects
that exist in composite structures with geometric nonlinearity.   LSFEs are
high-order finite elements with nodes located at the Gauss-Legendre-Lobatto
points.  LSFEs can be an order of magnitude more efficient that low-order
finite elements for a given accuracy level.    The LSFE code is implemented
in the software module called BeamDyn in the new FAST modularization
framework for dynamic simulation of highly flexible composite-material wind
turbine blades.  The framework allows for simulations of
wind turbines in operating conditions. In this paper, we verify BeamDyn
for static and dynamic nonlinear deformation of composite beams and compare
BeamDyn LSFE performance against common low-order finite elements found in a
commercial code.    Comparisons show that the BeamDyn LSFEs can provide
dramatically more accurate results for a given model size.
\end{abstract}

%\section*{Nomenclature}

 %\begin{tabbing}
  %XXX \= \kill % first line sets tab stop
  %$J$ \> Jacobian Matrix \\
  %$f$ \> Residual value vector \\
  %$x$ \> Variable value vector \\
  %$F$ \> Force, N \\
  %$m$ \> Mass, kg \\
  %$\Delta x$ \> Variable displacement vector \\
  %$\alpha$ \> Acceleration, m/s\textsuperscript{2} \\[5pt]
  %\textit{Subscript}\\
  %$i$ \> Variable number \\
 %\end{tabbing}

%&LaTeX
\section{Introduction} 

%Wind power is becoming one of the most important renewable energy sources in
%the United States as demonstrated by the fact that the electricity produced
%from wind amounted to 3.56\% of all generated electrical energy for the 12
%months until March 2013\cite{WindWiki}. 
Wind power is becoming one of the most important renewable-energy sources in
the United States.
%as demonstrated by the fact that the electricity produced
%from wind amounted to 3.56\% of all generated electrical energy for the 12
%months until March 2013\cite{WindWiki}. 
In recent years, the size of wind
turbines has been increasing to lower the cost, which leads
to highly flexible turbine blades. These huge electro-mechanical systems pose
a significant challenge for engineering design and analysis. Although
possible with modern super computers, direct three-dimensional (3D)
structural analysis is so computationally expensive that the wind industry is
always seeking efficient high-fidelity engineering models.

Beam models are widely used to represent and analyze engineering structures
that have one of its dimensions much larger than the other two. Many
engineering components can be idealized as beams: bridges in civil
engineering, joists and lever arms in heavy-machine industries, and
helicopter rotor blades. The blades, tower, and shaft in a wind turbine
system are well suited to idealization as beams. In the weight-critical
applications of beam structures, like high-aspect-ratio wings in aerospace
and wind energy, composite materials are attractive due to their superior
strength-to-weight and stiffness-to-weight ratios.  However, analysis of
composite-materials structures is more difficult than their isotropic
counterparts due to elastic-coupling effects. The geometrically exact beam
theory (GEBT), first proposed by Reissner\cite{Ressiner1973}, is a
method that has proven powerful for analysis of highly flexible composite
beams in the helicopter engineering community. During the past several
decades, much effort has been invested in this area. Simo\cite{Simo1985} and
Simo and Vu-Quoc\cite{Simo1986} extended Reissner's work to deal with 3D
dynamic problems. Jeleni\'c and Crisfield\cite{Crisfield1999} implemented
this theory using the finite-element (FE) method where a new approach for
interpolating the rotation field was proposed that preserves the geometric
exactness. Betsch and Steinmann\cite{Betsch2002} circumvented the
interpolation of rotation by introducing a re-parameterization of the weak
form corresponding to the equations of motion. It is noted that
Ibrahimbegovi\'c and his colleagues implemented this theory for
static\cite{Ibrahim1995} and dynamic\cite{Ibrahim1998} analysis. In contrast
to the displacement-based implementations, the geometrically exact beam theory
has also been formulated by mixed finite elements where both the primary and
dual fields are independently interpolated. In the mixed formulation, all of the necessary
ingredients, including Hamilton's principle and kinematic equations, are
combined in a single variational-formulation statement; Lagrange
multipliers, motion variables, generalized strains, forces and moments,
linear and angular momenta, and displacement and rotation variables are
considered as independent quantities. Yu et al.\cite{YuGEBT} and Wang et
al.\cite{Wang:GEBT2013}
presented the implementation of GEBT in a mixed formulation; various
rotation parameters were investigated and the code was validated against
analytical and numerical solutions. Readers are referred to
Hodges\cite{HodgesBeamBook}, where comprehensive derivations and discussions
on nonlinear composite-beam theories can be found.

Legendre spectral finite elements\cite{Patera:1984,Ronquist:1987} (LSFEs)
are $p$-type elements whose shape functions are Lagrangian
interpolants with node locations at the Gauss-Lobatto-Legendre (GLL) points.
LSFEs combine the accuracy of global spectral methods with the
geometric-modeling
flexibility of {\it h}-type FEs. LSFEs have seen successful use
in the simulation of fluid dynamics\cite{Ronquist:1987, Patera:1984,
Deville:2002}, two-dimensional elastic wave propagation in solid media in
geophysics \cite{Komatitsch:1998}, elastodynamics \cite{Sridhar:2006}, and
acoustic wave propagation \cite{Sprague:2004}. LSFEs have been applied to
the linear-response analysis of
beams\cite{Ben-Tal-etal:1995,Ben-Tal-etal:1996,Kudela-etal:2007a,Sprague-Geers:2008,Wang:SFE2013}
and plate elements\cite{Zrahia-Bar-Yoseph:1995,
Kudela-etal:2007b,Sprague-Brito:2012}. Xiao and Zhong \cite{Xiao-Zhong:2012} reported  a displacement-based implementation by LSFEs for two-dimensional static nonlinear beam deformation. Their
LSFEs were compared against a mixed-formulation
low-order-FE GEBT code by Wang and
Sprague\cite{Wang:SFE2013};  it was shown that the LSFEs provide
exponential convergence rates, while the low-order FEs were limited to
an algebraic convergence rate.

In this paper, we present a three-dimensional displacement-based
implementation of the geometrically exact beam theory using LSFEs.  This work
builds on previous efforts that showed the implementation of 3D rotation
parameters\cite{Wang:GEBT2013} and a demonstration example of
two-dimensional nonlinear spectral beam elements\cite{Wang:SFE2013} for
static deformation.  The code implemented in this work is in accordance with
the new FAST modularization framework\cite{Jonkman:2013},  which allows coupled aero-hydro-servo-elastic simulation of both land-based and offshore wind turbine under realistic operating conditions. The coupled code will be available in June 2014.

The paper is organized as follows.  The theoretical foundation of the
geometrically exact beam theory is introduced first. Then the spatial
discretization by LSFEs is discussed. Finally, verification examples are
provided to show the accuracy and efficiency of the present model  for
composite beams.  




~
\vspace{0.1in}
~
%&LaTeX
\section{Geometrically Exact Beam Theory}

For completeness, this section reviews the geometrically exact beam theory and
linearization process of the governing equations. The content of this
section can be found in many other papers and textbooks.
Figure~\ref{Kinematics} shows a beam in its initial undeformed and deformed
states. A reference frame $\mathbf{b}_i$, for $i=\{ 1,2,3 \}$, is introduced along the beam axis
for the undeformed state; a frame $\mathbf{B}_i$ is introduced along each
point of the deformed beam axis. Curvilinear-coordinate $x_1$ defines the
intrinsic parameterization of the reference line; \textcolor{red}{ and similarly, $s$ denotes the deformed reference line.}

\begin{figure}
\centering
\includegraphics[width=5.0in]{\directory Kinematics.eps}
\caption{Schematic of beam deformation} \label{Kinematics}
\end{figure}

In this paper, we use matrix notation to denote vectorial or vectorial-like
quantities. For example, we use a underline to denote a vector
$\underline{u}$, a bar to denote unit vector $\bar{n}$, and double underline
to denote a tensor $\underline{\underline{\Delta}}$. Note that sometimes the
underlines only denote the dimension of the corresponding matrix. The
governing equations of motion for geometrically exact beam theory can be written
as\cite{Bauchau:2010}
\begin{align}
	\label{GovernGEBT-1}
	\dot{\underline{h}} - \underline{F}^\prime &= \underline{f} \\
	\label{GovernGEBT-2}
	\dot{\underline{g}} + \dot{\tilde{u}} \underline{h} - \underline{M}^\prime - (\tilde{x}_0^\prime + \tilde{u}^\prime) \underline{F} &= \underline{m}
\end{align}
where $\vec{h}$ and $\vec{g}$ are the linear and angular momenta resolved in
the inertial coordinate system, respectively; $\vec{F}$ and $\vec{M}$ are
the beam's sectional forces and moments, respectively; $\vec{u}$ is the 1D
displacement of the reference line; $\vec{x}_0$ is the initial position vector of a
point along the beam's reference line; $\vec{f}$ and $\vec{m}$ are the
distributed force and moment applied to the beam structure.  
A prime 
indicates a derivative with respect to the beam axis
$x_1$ and an overdot indicates a derivative with respect to time. The
tilde operator,  i.e., $(\skew{\cdot})$, denotes a second-order, skew-symmetric
tensor corresponding to the given vector. In the literature, it is also
termed as "cross-product matrix". For example, for the vector
$\overline{n}$,
\[
	\skew{n} = 
	     		\begin{bmatrix}
			0 & -n_3 & n_2 \\
			n_3 & 0 & -n_1 \\
			-n_2 & n_1 & 0\\
			\end{bmatrix}	
\]
The constitutive equations relate the velocities to the momenta and the one-dimensional strain measures to the sectional resultants as
\begin{align}
	\label{ConstitutiveMass}
	\begin{Bmatrix}
	\underline{h} \\
	\underline{g}
	\end{Bmatrix}
	= \underline{\underline{\mathcal{M}}} \begin{Bmatrix}
	\dot{\underline{u}} \\
	\underline{\omega}
	\end{Bmatrix} \\
	\label{ConstitutiveStiff}
	\begin{Bmatrix}
	\underline{F} \\
	\underline{M}
	\end{Bmatrix}
	= \underline{\underline{\mathcal{C}}} \begin{Bmatrix}
	\underline{\epsilon} \\
	\underline{\kappa}
	\end{Bmatrix}
\end{align}
where $\underline{\underline{\mathcal{M}}}$ and
$\underline{\underline{\mathcal{C}}}$ are the $6 \times 6$ sectional mass
and stiffness matrices,respectively, note that they are not really tensors;
$\underline{\epsilon}$ and $\underline{\kappa}$ are the 1D strains and
curvatures, respectively. $\underline{\omega}$ is the angular velocity
vector that is defined by the rotation tensor $\underline{\underline{R}}$ as
$\underline{\omega} = \mathrm{axial}(\dot{\underline{\underline{R}}}~\underline{\underline{R}})$.

For a displacement-based finite element implementation, there are six
degree-of-freedoms (DoFs) at each node: 3 displacement components and 3
rotation components. Here we use $\vec{q}$ to denote the elemental
displacement array as $\underline{q}=\left[
\underline{u}^T~~\underline{p}^T\right]$ where $\vec{u}$ is the 1D
displacement and $\vec{p}$ is the rotation parameter vector. The
acceleration array can thus be defined as $\underline{a}=\left[
\ddot{\underline{u}}^T~~ \dot{\underline{\omega}}^T \right]$. For nonlinear
finite element analysis, the discretized and incremental forms of
displacement, velocity, and acceleration array are written as
\begin{align}
	\label{Discretized}
	\underline{q} (x_1) &= \underline{\underline{N}} ~\hat{\underline{q}}~~~~\Delta \underline{q}^T = \left[ \Delta \underline{u}^T~~\Delta \underline{p}^T \right] \\
	\underline{v}(x_1) &= \underline{\underline{N}}~\hat{\underline{v}}~~~~\Delta \underline{v}^T = \left[\Delta \underline{\dot{u}}^T~~\Delta \underline{\omega}^T \right] \\
%	\label{Incremental}
	\underline{a}(x_1) &= \underline{\underline{N}}~ \hat{\underline{a}}~~~~\Delta \underline{a}^T = \left[ \Delta \ddot{\underline{u}}^T~~\Delta \dot{\underline{\omega}}^T \right]	
\end{align}
where $\tens{N}$ is the shape function matrix and $(\hat{\bullet})$ denotes
a column matrix of nodal values. \textcolor{red}{It is noted that given the ``untensorial'' nature, we need to adopt some special algorithm to deal with the 3D rotations which will be introduced in the next section.}   The governing equations for beams are
highly nonlinear so that a linearization process is needed. According to
Bauchau\cite{Bauchau:2010}, the linearized governing equations in
Eq.~\eqref{GovernGEBT-1} and \eqref{GovernGEBT-2} are in the form of
\begin{equation}
	\label{LinearizedEqn}
	\hat{\underline{\underline{M}}} \Delta \hat{\underline{a}} +\hat{\underline{\underline{G}}} \Delta \hat{\underline{v}}+ \hat{\underline{\underline{K}}} \Delta \hat{\underline{q}} = \hat{\underline{F}}^{ext} - \hat{\underline{F}}
\end{equation} 
where the $\hat{\tens{M}}$, $\hat{\tens{G}}$, and $\hat{\tens{K}}$ are the
elemental mass, gyroscopic, and stiffness matrices, respectively;
$\hat{\vec{F}}$ and $\hat{\vec{F}}^{ext}$ are the elemental forces and
externally applied loads, respectively. They are defined as follows
\begin{align}
	\label{hatM} 
	\hat{\tens{M}}&= \int_0^l \underline{\underline{N}}^T \mathcal{\underline{\underline{M}}} ~\underline{\underline{N}} dx_1 \\
	\label{hatG}
	\hat{\tens{G}} &= \int_0^l \tens{N}^T \tens{\mathcal{G}}^I~\tens{N} dx_1\\ 
	\label{hatK}
	\hat{\tens{K}}&=\int_0^l \left[ \tens{N}^T (\tens{\mathcal{K}}^I + \mathcal{\tens{Q}})~ \tens{N} + \tens{N}^T \mathcal{\tens{P}}~ \tens{N}^\prime + \tens{N}^{\prime T} \mathcal{\tens{C}}~ \tens{N}^\prime + \tens{N}^{\prime T} \mathcal{\tens{O}}~ \tens{N} \right] d x_1 \\	
	\label{hatF}
	\hat{\vec{F}} &= \int_0^l (\tens{N}^T \vec{\mathcal{F}}^I + \tens{N}^T \mathcal{\vec{F}}^D + \tens{N}^{\prime T} \mathcal{\vec{F}}^C)dx_1 \\
	\label{hatFext}
	\hat{\vec{F}}^{ext}& = \int_0^l \tens{N}^T \mathcal{\vec{F}}^{ext} dx_1 
\end{align}
The new matrix notations in Eq.~\eqref{hatM} to \eqref{hatFext} are briefly
introduced here. $\mathcal{\tens{M}}$ is the sectional mass matrix resolved
in inertial system; $\mathcal{\vec{F}}^C$ and $\mathcal{\vec{F}}^D$ are
elastic forces obtained from Eq.~\eqref{GovernGEBT-1} and
\eqref{GovernGEBT-2} as
\begin{align}
	\label{FC}
	\mathcal{\vec{F}}^C &= \begin{Bmatrix}
         \vec{F} \\
	\vec{M}
	\end{Bmatrix} = \tens{\mathcal{C}} \begin{Bmatrix}
	\vec{\epsilon} \\
	\vec{\kappa}
	\end{Bmatrix} \\
	\label{FD}
	\mathcal{\vec{F}}^D & = \begin{bmatrix}
	\underline{\underline{0}} & \underline{\underline{0}}\\
	(\tilde{x}_0^\prime+\tilde{u}^\prime)^T & \underline{\underline{0}}
	\end{bmatrix}
	\mathcal{\vec{F}}^C \equiv \tens{\Upsilon}~ \mathcal{\vec{F}}^C
\end{align}
where $\underline{\underline{0}}$ denotes a $3 \times 3$ null matrix. The $\tens{\mathcal{G}}^I$, $\tens{\mathcal{K}}^I$,  $\mathcal{\tens{O}}$, $\mathcal{\tens{P}}$, $\mathcal{\tens{Q}}$, and $\vec{\mathcal{F}}^I$ in Eq.~\eqref{hatG}, Eq.~\eqref{hatK}, and Eq.~\eqref{hatF} are defined as
\begin{align}
        \label{mathcalG}
        \tens{\mathcal{G}}^I &= \begin{bmatrix}
        \tens{0} & (\tilde{\omega} m \vec{\eta})^T+\tilde{\omega} m \tilde{\eta}^T  \\
        \tens{0} & \tilde{\omega} \tens{\varrho}-\widetilde{\tens{\varrho} \vec{\omega}}
        \end{bmatrix} \\
        \label{mathcalK}
        \tens{\mathcal{K}}^I &= \begin{bmatrix}
        \tens{0} & \dot{\tilde{\omega}}m\tilde{\eta}^T + \tilde{\omega} \tilde{\omega}m\tilde{\eta}^T  \\
        \tens{0} & \ddot{\tilde{u}}m\tilde{\eta} + \tens{\varrho} \dot{\tilde{\omega}}-\widetilde{\tens{\varrho} \vec{\dot{\omega}}}+\tilde{\omega} \tens{\varrho} \tilde{\omega} - \tilde{\omega}  \widetilde{\tens{\varrho} \vec{\omega}}
        \end{bmatrix}\\
	\label{mathcalO}
	\mathcal{\tens{O}} &= \begin{bmatrix}
	\tens{0} & \tens{C}_{11} \tilde{E_1} - \tilde{F} \\
	\tens{0}& \tens{C}_{21} \tilde{E_1} - \tilde{M}
	\end{bmatrix} \\
	\label{mathcalP}
	\mathcal{\tens{P}} &= \begin{bmatrix}
	\tens{0} & \tens{0} \\
	\tilde{F} +  (\tens{C}_{11} \tilde{E_1})^T & (\tens{C}_{21} \tilde{E_1})^T
	\end{bmatrix}  \\
	\label{mathcalQ}
	\mathcal{\tens{Q}} &= \tens{\Upsilon}~ \mathcal{\tens{O}} \\
	\label{mathcalF}
	\vec{\mathcal{F}}^I &= \begin{Bmatrix}
	m \ddot{\vec{u}} + (\dot{\tilde{\omega}} + \tilde{\omega} \tilde{\omega})m \vec{\eta} \\
	m \tilde{\eta} \ddot{\vec{u}} +\tens{\varrho}\dot{\vec{\omega}}+\tilde{\omega}\tens{\varrho}\vec{\omega}
	\end{Bmatrix}
\end{align}
where the following notations were introduced to simplify the writing of the above expressions
\begin{align}
    \label{E1}
    \vec{E}_1 &= \vec{x}_0^\prime + \vec{u}^\prime \\
    \label{PartC}
    \tens{\mathcal{C}} &= \begin{bmatrix}
    \tens{C}_{11} & \tens{C}_{12} \\
    \tens{C}_{21} & \tens{C}_{22}
    \end{bmatrix}
\end{align} 
The derivation and linearization of governing equations of geometrically
exact beam theory can be found in Bauchau\cite{Bauchau:2010}.

It is pointed out that the three-dimensional rotations are represented by Wiener-Milenkovi\'c parameters \cite{Bauchau-etal:2008,Wang:GEBT2013} defined in the following equation:
 \begin{equation}
     \vec{p} = 4 \tan\left(\frac{\phi}{4} \right) \bar{n} 
     \label{WMParameter}
 \end{equation}
where $\phi$ is the rotation angle and $\bar{n}$ is the unit vector of
rotation axis. It can be observed that the valid range for this parameter is $|\phi| < 2 \pi$ where a singularity point will be reached at $2\pi$. The singularities existing at multiples of $\pm 2 \pi$ in the
above definition can be removed by a rescaling operation, as given in Ref~\cite{Bauchau-etal:2008}:
\begin{equation}
    \label{RescaledWM}
    \vec{r} = \begin{cases}
    4(q_0\vec{p} + p_0 \vec{q} + \tilde{p} \vec{q} ) / (\Delta_1 + \Delta_2), & \text{if } \Delta_2 \geq 0 \\
    -4(q_0\vec{p} + p_0 \vec{q} + \tilde{p} \vec{q} ) / (\Delta_1 - \Delta_2), & \text{if } \Delta_2 < 0
    \end{cases}
\end{equation}
where $\vec{p}$, $\vec{q}$, and $\vec{r}$ are the vectorial parameterization of three finite rotations such that $\tens{R}(\vec{r}) = \tens{R}(\vec{p}) \tens{R}(\vec{q})$; $p_0 = 2 - \vec{p}^T \vec{p}/8$, $q_0 = 2 - \vec{q}^T \vec{q}/8$, $\Delta_1 = (4-p_0)(4-q_0)$, and $\Delta_2 = p_0 q_0 - \vec{p}^T \vec{q}$.

\mas{Add another sentence about invariance}


%&LaTeX
\section{Numerical Implementation with Legendre Spectral Finite Elements}
The displacement fields in an element are approximated as
\begin{align}
    \label{InterpolateDisp}
    \vec{u}(\xi) &= \sum_{k=1}^{p+1} h^k(\xi) \vec{\hat{u}}^k \\
    \label{InterpolateDispp}
    \vec{u}^\prime(\xi) &= \sum_{k=1}^{p+1} h^{k\prime}(\xi) \vec{\hat{u}}^k
\end{align}
where $h^k(\xi)$, the component of shape function matrix $\tens{N}$, is the $p^{th}$-order polynomial
Lagrangian-interpolant shape function of node $k$, $k=\{1,2,...,p+1\}$, 
$\vec{\hat{u}}^k$ is
the $k^{th}$ nodal value, and $\xi \in \left[-1,1\right]$ is the element
natural coordinate.
However, as discussed in Bauchau et al.\cite{Bauchau-etal:2008}, the 
3D rotation field cannot simply be interpolated as the displacement field in the form of
\begin{align}
    \label{InterpolateRot}
    \vec{c}(\xi) &= \sum_{k=1}^{p+1} h^k(\xi) \vec{\hat{c}}^k \\
    \label{InterpolateRotp}
    \vec{c}^\prime(\xi) &= \sum_{k=1}^{p+1} h^{k \prime}(\xi) \vec{\hat{c}}^k 
\end{align}    
where $\vec{c}$ is the rotation field in an element and $\vec{\hat{c}}^k$ is
the nodal value at the $k^{th}$ node, for three reasons: 1) rotations do not
form a linear space so that they must be  ``composed'' rather than added; 2)
a rescaling operation is needed to eliminate the singularity existing in the
vectorial rotation parameters; 3) the rotation field lacks objectivity,
which, as
defined by Jeleni\'c and Crisfield\cite{Crisfield1999}, refers to the
invariance of strain measures computed through interpolation to the addition
of a rigid-body motion. Therefore, we adopt the more robust interpolation
approach proposed by Jeleni\'c and Crisfield\cite{Crisfield1999} to deal
with the finite rotations. Our approach is described as follows
\begin{description}

    \item[Step 1:] Compute the nodal relative rotations, $\vec{\hat{r}}^k$,
by removing the reference rotation, $\vec{\hat{c}}^1$, from the finite
rotation at each node, $\vec{\hat{r}}^k = (\vec{\hat{c}}^{1-}) \oplus
\vec{\hat{c}}^k$. It is noted that the minus sign on $\vec{\hat{c}}^1$ denotes that the relative rotation is calculated by removing the reference rotation from each node.  The composition in that equation is an equivalent of $\tens{R}(\vec{\hat{r}}^k) = \tens{R}^T(\vec{\hat{c}}^1)~\tens{R}(\vec{\vec{c}}^k).$

    \item[Step 2:] Interpolate the relative-rotation field: $\vec{r}(\xi) = h^k(\xi) \vec{\hat{r}}^k$ and $\vec{r}^\prime(\xi) = h^{k \prime}(\xi) \vec{\hat{r}}^k$. Find the curvature field $\vec{\kappa}(\xi) = \tens{R}(\vec{\hat{c}}^1) \tens{H}(\vec{r}) \vec{r}^\prime$, where $\tens{H}$ is the tangent tensor that relates the curvature vector $\vec{k}$ and rotation vector $\vec{p}$ as
\begin{equation}
    \label{Tensor}
    \vec{k} = \tens{H}~ \vec{p}^\prime
\end{equation}

    \item[Step 3:] Restore the rigid-body rotation removed in Step 1: $\vec{c}(\xi) = \vec{\hat{c}}^1 \oplus \vec{r}(\xi)$.
\end{description} 


Note that the relative-rotation field can be computed with respect to any of
the nodes of the element; we choose node 1 as the reference node for
convenience. In the LSFE approach, shape functions (i.e., those composing $\tens{N}$) are
$p^{th}$-order Lagrangian interpolants, where nodes are located at the $p+1$
GLL-quadrature points in the $[-1,1]$ element natural-coordinate domain.
Figure~\ref{fig:N4_lsfe} shows representative LSFE basis functions for  
fourth- and eighth-order elements.  Note that nodes are clustered near
element endpoints.
In the present implementation, weak-form integrals are evaluated with
$p$-point reduced Gauss quadrature.

\begin{figure}[h]
    \centering
    \psfrag{x}[][]{$\xi$}
   \subfloat[$p=4$]{
   \includegraphics[width=0.3\textwidth,clip=true]{\directory N4.eps}}
   \subfloat[$p=8$]{
   \includegraphics[width=0.3\textwidth,clip=true]{\directory N8.eps}}
    \caption{Representative $p+1$ Lagrangian-interpolant shape functions in
the element natural coordinates for
(a) fourth- and (b) eighth-order LSFEs, where nodes are located at the
Gauss-Lobatto-Legendre points.}
    \label{fig:N4_lsfe}
\end{figure}

The geometrically exact beam theory has been implemented with LSFEs in a
code called BeamDyn, which is a new module of FAST for wind turbine analysis. The system of nonlinear
equations in Eqs.~\eqref{GovernGEBT-1} and \eqref{GovernGEBT-2} are solved
using the Newton-Raphson method with the linearized form in
Eq.~\eqref{LinearizedEqn}. 
In the present
implementation, an energy-like stopping criterion has been chosen, which is calculated as
\begin{equation}
    \label{StoppingCriterion}
    \| \Delta \mathbf{U}^{(i)T} \left( \fourIdx{t+\Delta t}{}{}{}{\mathbf{R}} -  \fourIdx{t+\Delta t}{}{(i-1)}{}{\mathbf{F}}  \right) \| \leq \| \epsilon_E \left( \Delta \mathbf{U}^{(1)T} \left( \fourIdx{t+\Delta t}{}{}{}{\mathbf{R}} - \fourIdx{t}{}{}{}{\mathbf{F}} \right) \right) \|
\end{equation}
where $\|\cdot\|$ denotes the Euclidean norm, $\Delta \mathbf{U}$ is the
incremental displacement vector, $\mathbf{R}$ is the vector of externally
applied nodal point loads, $\mathbf{F}$ is the vector of nodal point forces
corresponding to the internal element stresses, and $\epsilon_E$ is the
preset energy tolerance. The superscript on the left side of a variable
denotes the time-step number (in a dynamic analysis), while the one on the
right side denotes the Newton-Raphson iteration number. As pointed out by
Bathe and Cimento\cite{Bathe-Cimento:1980}, this criterion provides
a measure of when both the displacements and the forces are near their
equilibrium values. Time integration is performed using the
generalized-$\alpha$ scheme in BeamDyn, which is an unconditionally stable
(for linear systems),
second-order accurate algorithm.  The scheme allows for users to choose
integration parameters that introduce high-frequency numerical dissipation.
More details
regarding the generalized-$\alpha$ method can be found in
Refs.\cite{Chung-Hulbert:1993,Bauchau:2010}. 



%%&LaTeX
\section{Numerical Implementation}

\mas{Describe numerical implementation as BeamDyn}

\mas{Describe NewtonRaphson solve (concisely) and stopping criterion.}

\mas{Describe time integration scheme, and some of the specific parameters
chosen. Mention numerical dissipation.}

\mas{Describe briefly Dymore -- order of elements, type of rotation
parameters, and time-integration scheme}





\section{Numerical Examples}

%&LaTeX
\subsection{Example 1: Static bending of a cantilever beam}

The first example is a common benchmark problem for geometrically nonlinear
analysis of beams\cite{Simo1985,Xiao-Zhong:2012}. We calculate the static
deflection of a cantilever beam that is subjected at its free end to
a constant moment about the $x_2$ axis, $M_2$; a system schematic is shown in Figure~\ref{E1Sketch}.  The length of the beam $L$ is $10$ inches and the cross-sectional stiffness 
matrix is 
\begin{equation}
    \label{StifE1}
    C^* = 10^3 \times \begin{bmatrix}
	1770 & 0    & 0    & 0    & 0    & 0   \\
	 0    & 1770 & 0    & 0    & 0    & 0   \\
	 0   &   0   & 1770 & 0    & 0    & 0   \\
	 0   &   0   &  0    & 8.16 & 0    & 0   \\
	 0   &   0   &  0    &  0    & 86.9 & 0   \\
	 0   &   0   &  0    &  0    &   0   & 215
\end{bmatrix}
\end{equation}
which has units of $C_{ij}^*$ (lb), $C_{i,j+3}^*$ (lb.in), and
$C_{i+3,j+3}^*$ (lb.in$^2$) for $i,j = 1,2,3$; these units apply to all
subsequent stiffness matrices. It is pointed out that the term with an
asterisk denotes that it is resolved in the material coordinate system. 

\begin{figure}
    \centering \includegraphics[width=0.5\textwidth]{\directory
E1Sketch.eps} \caption{Schematic of a cantilever beam with tip moment,
which was used in BeamDyn verification and performance studies.}
    \label{E1Sketch}
\end{figure} 

The load applied at the tip is given by 
\begin{equation}
    \label{E1Load}
    M_2 = \lambda \bar{M}_2
\end{equation}
where $\bar{M}_2 = \pi \frac{EI_2}{L}$; and the parameter $\lambda$ will vary between $0$ and $2$. In 
this case, the beam is discretized with two $5^{th}$-order Legendre
spectral FEs. The
static deformations of the beam obtained from BeamDyn are shown in
Figure~\ref{E1Deform} for six different tip moments.
The calculated tip displacements are compared with the analytical solution,
which can be found in Mayo et al.\cite{Mayo-etal:2004} as
\begin{equation}
    \label{E1Analytical}
    u_1 = \rho \sin \left( \frac{x_1}{\rho} \right) - x_1~~~~~u_3 = \rho
\left(1-\cos\left(\frac{x_1}{\rho}\right) \right)
\end{equation}
Analytical and BeamDyn-calculated results can be found in Table~\ref{E1u1}
and \ref{E1u3}. At this discretization level, BeamDyn results are virtually
identical to those of the analytical solution.

\begin{figure}
    \centering
    \includegraphics[width=0.5\textwidth]{\directory E1Deform.eps}

    \caption{Static deflection of a cantilever beam under six constant
bending moments as calculated with two 5$^{th}$-order Legendre spectral
FEs in BeamDyn.}

    \label{E1Deform}
\end{figure}

\begin{table}[tbp]
\centering 
\caption{Comparison of analytical and BeamDyn-calculated tip axial displacement
$u_1$ of a cantilever beam subject to a constant moment (in inches); the BeamDyn
model was composed of two 5$^{th}$-order LSFEs.}
\label{E1u1} 
	\begin{tabular}{| l | l | l | }
    	\hline
    	$\lambda$ & Analytical & BeamDyn  \\ \hline
    	0.4       & -2.4317    & -2.4317  \\ \hline
    	0.8       & -7.6613    & -7.6613  \\ \hline
    	1.2       & -11.5591   & -11.5591 \\ \hline
    	1.6       & -11.8921   & -11.8921 \\ \hline
    	2.0       & -10.0000   & -10.0000 \\ \hline
    \end{tabular}
\end{table}

\begin{table}[tbp]
\centering 
\caption{Comparison of analytical and BeamDyn-calculated tip vertical displacement
$u_3$ of a cantilever beam subject to a constant moment (in inches); the BeamDyn
model was composed of two 5$^{th}$-order LSFEs.}
\label{E1u3} 
	\begin{tabular}{| l | l | l | }
    	\hline
    	$\lambda$ & Analytical & BeamDyn  \\ \hline
    	0.4       & 5.4987     & 5.4987   \\ \hline
    	0.8       & 7.1978     & 7.1979   \\ \hline
    	1.2       & 4.7986     & 4.7986   \\ \hline
    	1.6       & 1.3747     & 1.3747   \\ \hline
    	2.0       & 0.0000     & 0.0000   \\ \hline
    \end{tabular}
 \end{table}

The rotation parameter $p_2$ at each node along beam axis $x_1$ obtained from BeamDyn are plotted in Figure~\ref{E1Rot:Rot} for $\lambda = 0.8$ and $\lambda = 2.0$, respectively. A rescaling can be observed from this figure for the case $\lambda = 2.0$. It is noted that although the rotation parameters are not continuous between elements due to the rescaling operation, the relatively rotations are continuous in a single element as described in the previous section, which can be observed from Figure~\ref{E1Rot:Rel}.  
\begin{figure}
    \centering
    \begin{tabular}{c}
    \subfloat[Rotation parameter $p_2$]{\label{E1Rot:Rot}\includegraphics[width=3.0in]{\directory E1Rot.eps}} \qquad 
    \subfloat[Relative rotation $r_2$]{\label{E1Rot:Rel}\includegraphics[width=3.0in]{\directory E1RelRot.eps}} 
    \end{tabular}
    \caption{(a) Wiener-Milenkovi\'c rotation parameters along beam axis $x_1$
as calculated by BeamDyn for two tip moments; (b) Relative rotations in two elements for the case $\lambda = 2.0$.  }
    \label{E1Rot}
\end{figure}

Finally, we conduct a convergence study of the BeamDyn LSFEs. The
convergence rate is compared with conventional quadratic elements used in
Dymore \cite{Dymore:2013}, which is a finite-element
based multibody dynamics code for the comprehensive modeling of flexible
multibody systems. 
%It is pointed out that user can choose element up to
%$3^rd$ order in Dymore. 
Figure~\ref{E1Conv} shows the normalized error
$\varepsilon(u)$, where $u$ is the calculated tip displacement (at $x=L$), as a function
of the number of model nodes for the calculation with Dymore quadratic finite
elements (QFE) and a single Legendre spectral element finite (LSFE), where
\begin{equation}
    \label{E1Error}
    \varepsilon(u) = \left| \frac{u-u^a}{u^a} \right|
\end{equation}
and where $u^a$ is the analytical solution.  The parameter $\lambda$ is set
to $1.0$ for this case. The Legendre spectral elements (with $p$-refinement)
exhibit highly desirable exponential convergence to machine-precision error,
whereas the conventional quadratic elements are limited to algebraic
convergence.    For a given model size, an LSFE model can be orders of magnitude
more accurate than its QFE counterpart.

\begin{figure}
    \centering
    \psfrag{xaxis}[][]{Number of Nodes}
    \begin{tabular}{c}
    \subfloat[$u_1$]{\label{E1Conv:u1}\includegraphics[width=3.0 in]{\directory  E1Convu1.eps}} \qquad
\subfloat[$u_3$]{\label{E1Conv:u3}\includegraphics[width=3.0in]{\directory  E1Convu3.eps}}\\
\end{tabular}
\caption{Normalized error of the (a) $u_1$ and (b) $u_3$ tip displacements
of a cantiler beam (Figure~\ref{E1Sketch}) under constant tip
moment ($\lambda = 1.0$) as a function of the total number of nodes. Results were calculated
with BeamDyn (LSFE) and Dymore (QFE).  LSFE model refinement was accomplished
by increasing polynomial order and QFE model refinement was accomplished by
increasing the number of elements.  }
\label{E1Conv}
\end{figure}



%&LaTeX
\subsection{Example 2: Static analysis of a composite beam}
The second example is to show the capability of BeamDyn for composite beams
with elastic couplings. The cantilever beam used in this case is $10$ inches
long with a boxed cross-section made of composite materials that can be
found in Yu et al.\cite{Yu-etal:2002}. Readers are referred to
Figure~\ref{E1Sketch} for a schematic of this example system. The stiffness matrix is given as
\begin{equation}
C^* = 10^3 \times \begin{bmatrix}
	1368.17 & 0     & 0     & 0      & 0      & 0      \\
	0       & 88.56 & 0     & 0      & 0      & 0      \\
	0       & 0     & 38.78 & 0      & 0      & 0      \\
	0       & 0     & 0     & 16.96  & 17.61  & -0.351 \\
	0       & 0     & 0     & 17.61  & 59.12  & -0.370 \\
	0       & 0     & 0     & -0.351 & -0.370 & 141.47
\end{bmatrix}
\label{E2Stif}
\end{equation}
A concentrated force $P = 150~lbs$ along the $x_3$ direction is applied at
the free tip. In the BeamDyn analysis, the beam is meshed with two
$5^{th}$-order elements. The displacements and rotation parameters at each
node along beam axis are plotted in Figure~\ref{E2U}.  It is noted that the
coupling effects exist between twist and two bendings. The applied in-plane
force leads to a fairly large twist angle due to the bending-twist coupling,
which can be observed in Figure~\ref{E2Rot}. 
%It is also noted that the
%internal nodes of Legendre Spectral Finite Elements are not evenly placed,
%which is different from conventional elements. 

\begin{figure}
    \centering
    \begin{tabular}{c}
    \subfloat[$Displacements$]{\label{E2Disp}\includegraphics[width=3.0 in]{\directory  E2Disp.eps}} \qquad
\subfloat[$Rotations$]{\label{E2Rot}\includegraphics[width=3.0in]{\directory  E2Rot.eps}}\\
\end{tabular}
\caption{Displacements and rotation parameters along beam axis for Example 2.}
\label{E2U}
\end{figure}

The tip displacements and rotations are compared with those obtained by Dymore in Table~\ref{E2Tip} for verification, where the beam is meshed with 10 $3^{rd}$ order elements. Good agreement can be observed between BeamDyn and Dymore results.
\begin{table}[tbp]
\centering 
\caption{Numerically determined tip displacements and rotation parameters of
a composite beam in Example 2 as calculated by BeamDyn (LSFEs) and Dymore
(QEs) }
\label{E2Tip} 
	\begin{tabular}{| l | l | l | l | l | l | l |}
    	\hline
    	 & $u_1$ (inch)& $u_2$(inch) & $u_3$(inch)  & $p_1$ & $p_2$ & $p_3$  \\ 
	 \hline
	 BeamDyn & -0.09064 & -0.06484 & 1.22998 & 0.18445 & -0.17985 & 0.00488 \\
	 \hline
	 Dymore & -0.09064 & -0.06483 & 1.22999 & 0.18443 & -0.17985 & 0.00488 \\
    	\hline
    \end{tabular}
\end{table}



%&LaTeX
\subsection{Example 3: Dynamic analysis of a composite beam under sinusoidal force at the tip}

The last example is a transient analysis of a composite beam with boxed
cross-section; the beam has the same
geometry and boundary conditions as that of the previous example. The mass
sectional properties are given by VABS \cite{Yu-etal:2002,Wang-Yu:2012} as
\begin{equation}
    \label{E3Mass}
    M^* = 10^{-2} \times \begin{bmatrix}
	8.538 & 0     & 0     & 0      & 0      & 0      \\
	0       & 8.538 & 0     & 0      & 0      & 0      \\
	0       & 0     & 8.538 & 0      & 0      & 0      \\
	0       & 0     & 0     & 1.4433  & 0  & 01 \\
	0       & 0     & 0     & 0  & 0.40972  &0 \\
	0       & 0     & 0     & 0 & 0 & 1.0336
\end{bmatrix}    
\end{equation}
The units associated with the mass matrix values are $M_{ii}^*$ (lb s$^2$/in$^2$) and $M_{i+3,i+3}^*$ (lb s$^2$) for $i = 1,2,3$. The beam is divided into two $5^{th}$-order elements in the current calculation and a sinusoidal point dead force is applied at the free tip in the $x_3$ direction given as
\begin{equation}
    \label{E3AppliedForce}
    P_3 = A_F~\text{sin}(\omega_F~t)
\end{equation}
where $A_F = 1.0 \times 10^2$ lbs and $\omega_F = 10$ rad/s (see
Figure~\ref{E3SinForce}). 
The spectral radius $\rho_\infty$ is set to $0.0$ in the time integrator so that high frequency numerical dissipation can be achieved. The tip displacement and rotation histories of the beam are plotted in
Figure~\ref{E3U}, where the time step was $0.005$ s. Note that all of the components, including three displacements and three rotations, are non-zero due to the elastic-coupling effects. The time histories of the stress resultants at the root of the beam are given in Figure~\ref{E3F}.
\begin{figure}
    \centering
    \psfrag{Time}{blah}
    \includegraphics[width=3.0in]{\directory E3SinForce.eps}
    \caption{The applied sinusoidal vertical force at the tip in Example 3 .}
    \label{E3SinForce}
\end{figure}

\begin{figure}
    \centering
    \begin{tabular}{c}
    \subfloat[$u_1$]{\label{E3U:u1}\includegraphics[width=3.0 in]{\directory  E3Tipu1.eps}} \qquad
\subfloat[$u_2$]{\label{E3U:u2}\includegraphics[width=3.0in]{\directory  E3Tipu2.eps}} \\
\subfloat[$u_3$]{\label{E3U:u3}\includegraphics[width=3.0 in]{\directory  E3Tipu3.eps}} \qquad
\subfloat[$p_1$]{\label{E3U:p1}\includegraphics[width=3.0 in]{\directory  E3Tipp1.eps}} \\
\subfloat[$p_2$]{\label{E3U:p2}\includegraphics[width=3.0 in]{\directory  E3Tipp2.eps}} \qquad
\subfloat[$p_3$]{\label{E3U:p3}\includegraphics[width=3.0 in]{\directory  E3Tipp3.eps}} \\
\end{tabular}
\caption{Tip displacement and rotation histories of a composite beam under vertical load.}
\label{E3U}
\end{figure} 

\begin{figure}
    \centering
    \begin{tabular}{c}
    \subfloat[$F_1$]{\label{E3F:F1}\includegraphics[width=3.0 in]{\directory  E3RootF1.eps}} \qquad
\subfloat[$F_2$]{\label{E3F:F2}\includegraphics[width=3.0in]{\directory  E3RootF2.eps}} \\
\subfloat[$F_3$]{\label{E3F:F3}\includegraphics[width=3.0 in]{\directory E3RootF3.eps}} \qquad
\subfloat[$M_1$]{\label{E3F:M1}\includegraphics[width=3.0 in]{\directory  E3RootM1.eps}} \\
\subfloat[$M_2$]{\label{E3F:M2}\includegraphics[width=3.0 in]{\directory  E3RootM2.eps}} \qquad
\subfloat[$M_3$]{\label{E3F:M3}\includegraphics[width=3.0 in]{\directory  E3RootM3.eps}} \\
\end{tabular}
\caption{Stress resultant time histories at the root of a composite beam.}
\label{E3F}
\end{figure} 
Finally, we examine here the convergence rates of the LSFEs and conventional
quadratic elements (in Dymore). Figure~\ref{E3Conv} shows normalized root-mean-square (RMS) error of the numerical solutions for the displacement $u_1$ at the free tip over the time interval $0 \leq t  \leq 4$. Normalized RMS error for $n_{max}$ numerical response values $u_1^n$, where $u_1^n \approx u_1(t^n)$, was calculated as
\begin{equation}
    \label{RMS}
    \varepsilon_{\mathrm{RMS}}(u_1) = \sqrt{\frac{\sum_{k=0}^{n_{max}} \left[ u_1^k - u_b(t^k) \right]^2}{\sum_{k=0}^{n_{max}} \left[ u_b(t^k) \right]^2}}
\end{equation}
where $u_b(t)$ is the benchmark solution; here $u_b(t)$ is a highly resolved
numerical solution obtained by BeamDyn with one $20^{th}$-order element and
the time increment was $\Delta t_b = 1.0 \times 10^{-4}$ s. Two time-increment sizes
are examined in the test calculations: $\Delta t_1 = 5.0 \times 10^{-3}$ s and
$\Delta t_2 = \frac{\Delta t_1}{2}$. The following observations can be made from Figure~\ref{E3Conv}:
\begin{itemize}

    \item For a fixed $\Delta t$, both Dymore (QFEs) and BeamDyn (LSFEs)
converge with spatial refinement to the same error level. BeamDyn is
converged with only five nodes, whereas Dymore requires at least nine nodes.

    \item  The converged error levels are due exclusively to
time-discretization error.  We note that the converged error for $\Delta t_2
= \Delta t_1/2$ is one-fourth that for $\Delta t_1$, which is expected for our
second-order-accurate time integrator.  

%The non-zero errors are in the time integration scheme. By reducing time
%increment step $\Delta t$ by 2 ($\Delta t_2 = \frac{\Delta t_1}{2}$), the
%error is reduced by $4$ ($\varepsilon_2 = \frac{\varepsilon_1}{4}$), which
%is expected for a second order accurate time integrator.

%    \item The convergence rate of LSFE in the space domain is exponential
%    as %expected, which is much faster than the conventional quadratic finite elements.
\end{itemize}  

%It can be observed that \textcolor{red}{2 observations:1 error due to time integrator; 2 better convergence rate of LSFE.}
\begin{figure}
    \centering
    \includegraphics[width = 3.0 in]{\directory E3u1Conv.eps}
    \caption{Normalized RMS error of tip displacement $u_1$ histories over
$0 \leq t \leq 4$ as a function of number of nodes as calculated by BeamDyn
(LSFEs) and Dymore (QFEs).}
    \label{E3Conv}
\end{figure}
 


%&LaTeX
\section{Conclusion}

This is the conclusion.



\section*{Acknowledgments} 

This work was supported by the U.S. Department of Energy under Contract No.\
DE-AC36-08-GO28308 with the National Renewable Energy Laboratory. Support
was partially provided through a Laboratory Directed Research and Development grant
\textit{High-Fidelity Computational Modeling of Wind-Turbine Structural
Dynamics}. The authors acknowledge Professor Oliver A.  Bauchau for the technical
discussions on the 3D rotation parameters.
 

\bibliographystyle{aiaa}
\bibliography{references,local_bib}

\end{document}
