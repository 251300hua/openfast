\documentclass{aiaa-tc}

\usepackage{color}
\usepackage{amsmath}
%\usepackage{overcite}
\usepackage{graphicx}
\usepackage{subfig}
\usepackage{authblk}


\input basic.ltx
\def\directory{EPSF/}

%------------------------------------------------------------------------------
% MAS Additions
\newcommand{\mas}[1]{\textcolor{magenta}{#1}}
\newcommand{\qw}[1]{\textcolor{blue}{#1}}
%useful for showing deleted text
\renewcommand{\kill}[1]{\textcolor{red}{\sout{#1}}}                             

\usepackage[normalem]{ulem}  % mas addition
\newcommand{\uvec}[1]{\bar{#1}}
\newcommand{\tens}[1]{\underline{\underline{#1}}}
\renewcommand{\vec}[1]{\underline{#1}}
\renewcommand{\skew}[1]{\widetilde{#1}}
%------------------------------------------------------------------------------

%\title{A Efficient High-Fidelity Beam Solver in FAST Modularization Framework}

\title{Nonlinear Legendre Spectral Finite Elements for Wind Turbine Blade Dynamics}
%{Rotation parameterization for analysis of geometrically nonlinear deformation of beams}

%{Structural Dynamics Analysis of Geometric Nonlinear Composite Beams}

% typeset with pdflatex, bibtex, pdflatex, pdflatex


%\author{Qi Wang%
 %        \thanks{Postdoctoral Research Fellow, AIAA Member.}\\
 %        \normalsize\itshape
 %        National Renewable Energy Laboratory, Golden, CO 80020}



 \author[1]{Qi Wang\thanks{Research Engineer, National Wind Technology Center, AIAA Member. Email: Qi.Wang2@nrel.gov}}
 \author[1]{Michael A. Sprague\thanks{Senior Research Scientist, Scientific Computing Center.}}
 \author[1]{Jason Jonkman\thanks{Senior Engineer, National Wind Technology Center.}}
 \author[2]{Nick Johnson\thanks{Graduate Research Assistant at Colorado School of Mines.}}
 \affil[1]{National Renewable Energy Laboratory, Golden, CO 80401}
 \affil[2]{Department of Mechanical Engineering, Colorado School of Mines, Golden, CO 80401}
 
 \renewcommand\Authands{ and }

\begin{document}

\maketitle

\begin{abstract}
This paper presents a numerical implementation and examination of new wind
turbine blade finite element model based on Geometrically Exact Beam Theory
(GEBT) and a high-order spectral finite element method.  The
displacement-based GEBT is presented, which includes the coupling effects
that exist in composite structures and geometric nonlinearity.   Legendre
spectral finite elements (LSFEs) are high-rder finite elements with nodes
located at the Gauss-Legendre-Lobatto points.  LSFEs can be an order of
magnitude more efficient that low-order finite elements for a given accuracy
level.    The new LSFE code is implemented in the new FAST Modularization Framework for
dynamic simulation of highly flexible composite-material wind turbine
blades.  The framework allows for fully interactive simulations of turbine
blades in operating conditions.  Numerical examples showing validation and
LSFE performance are provided in the numerical examples section. It concludes that the implemented code can be used as a efficient high-fidelity beam tool in FAST.
\end{abstract}

%\section*{Nomenclature}

 %\begin{tabbing}
  %XXX \= \kill % first line sets tab stop
  %$J$ \> Jacobian Matrix \\
  %$f$ \> Residual value vector \\
  %$x$ \> Variable value vector \\
  %$F$ \> Force, N \\
  %$m$ \> Mass, kg \\
  %$\Delta x$ \> Variable displacement vector \\
  %$\alpha$ \> Acceleration, m/s\textsuperscript{2} \\[5pt]
  %\textit{Subscript}\\
  %$i$ \> Variable number \\
 %\end{tabbing}

\section{Introduction} 

Wind power is becoming one of the most important renewable energy sources in
the United States as demonstrated by the fact that the electricity produced
from wind amounted to 3.56\% of all generated electrical energy for the 12
months until March 2013\cite{WindWiki}. In recent years, the size of wind
turbines has been increasing immensely to lower the cost, which also leads
to highly flexible turbine blades. This huge electro-mechanical system poses
a significant challenge for engineering design and analysis. Although
possible with modern super computers, direct three-dimensional (3D)
structural analysis is so computationally expensive that engineers are
always seeking for efficient high-fidelity simplified models.

Beam models are widely used to represent and analyze engineering structures
that have one of its dimensions much larger than the other two. Many
engineering components can be idealized as beams: bridges in civil
engineering, joists and lever arms in heavy-machine industries, and
helicopter rotor blades. The blades, tower, and shaft in a wind turbine
system can be considered as beams. In the weight-critical applications of
beam structures, like high-aspect-ratio wings in aerospace and wind energy,
composite materials are attractive due to their superior weight-to-strength
and weight-to stiffness ratios.  However, analysis of structures made of
composite materials is more difficult than their isotropic counterparts due
to the elastic coupling effects. The Geometrically Exact Beam Theory (GEBT), which
was first proposed by Reissner in 1973 \cite{Ressiner1973}, is a method that
has proven powerful for analysis of highly flexible composite beams in the
helicopter engineering community. During the past several decades, much
effort has been invested in this area. Simo\cite{Simo1985} and Simo and
Vu-Quoc\cite{Simo1986} extended Reissner's work to deal with
three-dimensional (3D) dynamic problems. Jeleni\'c and
Crisfield\cite{Crisfield1999} implemented this theory using the finite
element method where a new approach for interpolating the rotation field was
proposed that preserves the geometric exactness. Betsch and
Steinmann\cite{Betsch2002} circumvented the interpolation of rotation by
introducing a re-parameterization of the weak form corresponding to the
equations of motion of GEBT. It is noted that Ibrahimbegovi\'c and his
colleagues implemented this theory for static\cite{Ibrahim1995} and
dynamic\cite{Ibrahim1998} analysis. In contrast to the displacement-based
implementations, the geometric exact beam theory has also been formulated by
mixed finite elements where both the primary and dual field are
independently interpolated\cite{CookFEM}. In the mixed formulation, all of
the necessary ingredients, including Hamilton's principle and kinematic
equations, are combined in a single variational formulation statement;
Lagrange multipliers, motion variables, generalized strains, forces
and moments, linear and angular momenta, and displacement and rotation
variables are considered as independent quantities. Yu et al.\cite{YuGEBT,
Wang:GEBT2013} presented the implementation of GEBT in a mixed formulation;
various rotation parameters were investigated and the code was validated
against analytical and numerical solutions. The readers are referred to a
textbook by Hodges\cite{HodgesBeamBook}, where comprehensive derivations
and discussions on nonlinear composite beam theories can be found.

Legendre spectral finite elements\cite{Patera:1984,Ronquist:1987} are
$p$-type finite elements whose shape functions are Lagrangian interpolants
with node locations at the Gauss-Lobatto-Legendre points. It combines the
accuracy of global spectral methods with geometric flexibility of {\it
h}-type FEs. The spectral FEs have seen extensive, highly successful use in
the simulation of fluid dynamics\cite{Ronquist:1987, Patera:1984,
Deville:2002}, two-dimensional elastic wave propagation in solid media in
geophysics \cite{Komatitsch:1998}, elastodynamics \cite{Sridhar:2006}, and
acoustic wave propagation \cite{Sprague:2004}. However, it has seen limited
application to dynamic analysis of
beam\cite{Ben-Tal-etal:1995,Ben-Tal-etal:1996,Kudela-etal:2007a,Sprague-Geers:2008}
and plate
elements\cite{Zrahia-Bar-Yoseph:1995, Kudela-etal:2007b,Sprague-Brito:2012}. 

In this paper, we present a displacement-based implementation of
geometrically exact beam theory using Legendre spectral finite elements
(LSFEs).  This work builds on a previous effort which showed the
implementation of three-dimensional rotation parameters\cite{Wang:GEBT2013}
and a demonstration example of two-dimensional nonlinear spectral beam
elements\cite{Wang:SFE2013}. The theoretical foundation, the geometrically
exact beam theory, is introduced first. Then the interpolation of the theory by LSFEs is discussed. Finally, validation examples are provided to show the accuracy and
efficiency of the present model for composite beam accounting for elastic coupling effects. The code implemented in this work is in accordance to FAST Modularization Framework
\cite{Jonkman:2013},  which allows simulation of a whole turbine under
realistic operating conditions.

\section{Geometrically Exact Beam Theory}

For completeness, this section reviews the geometrically exact beam theory and
linearization process of the governing equations. The content of this
section can be found in many other papers and textbooks.
Figure~\ref{Kinematics} shows a beam in its initial undeformed and deformed
states. A reference frame $\mathbf{b}_i$ is introduced along the beam axis
for the undeformed state; a frame $\mathbf{B}_i$ is introduced along each
point of the deformed beam axis. Curvilinear coordinate $x_1$ defines the
intrinsic parameterization of the reference line.
\begin{figure}
\centering
\includegraphics[width=5.0in]{\directory Kinematics.eps}
\caption{Schematic of beam deformation} \label{Kinematics}
\end{figure}
In this paper, we use matrix notation to denote vectorial or vectorial-like
quantities. For example, we use a underline to denote a vector
$\underline{u}$, a bar to denote unit vector $\bar{n}$, and double underline
to denote a tensor $\underline{\underline{\Delta}}$. Note that sometimes the
underlines only denote the dimension of the corresponding matrix. The
governing equations of motion for geometric exact beam theory can be written
as\cite{Bauchau:2010}
\begin{align}
	\label{GovernGEBT-1}
	\dot{\underline{h}} - \underline{F}^\prime &= \underline{f} \\
	\label{GovernGEBT-2}
	\dot{\underline{g}} + \dot{\tilde{u}} \underline{h} - \underline{M}^\prime - (\tilde{x}_0^\prime + \tilde{u}^\prime) \underline{F} &= \underline{m}
\end{align}
where $\vec{h}$ and $\vec{g}$ are the linear and angular momenta resolved in
the inertial coordinate system, respectively; $\vec{F}$ and $\vec{M}$ are
the beam's sectional forces and moments, respectively; $\vec{u}$ is the 1D
displacement of the reference line; $\vec{x}_0$ is the initial position vector of a
point along the beam's reference line; $\vec{f}$ and $\vec{m}$ are the
distributed force and moment applied to the beam structure.  Notation
$(\bullet)^\prime$ indicates a derivative with respect to the beam axis
$x_1$ and $\dot{(\bullet)}$ indicates a derivative with respect to time. The
tilde operator $(\skew{\bullet})$ defines a second-order, skew-symmetric
tensor corresponding to the given vector. In the literature, it is also
termed as "cross-product matrix". For example,
\[
	\skew{n} = 
	     		\begin{bmatrix}
			0 & -n_3 & n_2 \\
			n_3 & 0 & -n_1 \\
			-n_2 & n_1 & 0\\
			\end{bmatrix}	
\]
The constitutive equations relate the velocities to the momenta and the one-dimensional strain measures to the sectional resultants as
\begin{align}
	\label{ConstitutiveMass}
	\begin{Bmatrix}
	\underline{h} \\
	\underline{g}
	\end{Bmatrix}
	= \underline{\underline{\mathcal{M}}} \begin{Bmatrix}
	\dot{\underline{u}} \\
	\underline{\omega}
	\end{Bmatrix} \\
	\label{ConstitutiveStiff}
	\begin{Bmatrix}
	\underline{F} \\
	\underline{M}
	\end{Bmatrix}
	= \underline{\underline{\mathcal{C}}} \begin{Bmatrix}
	\underline{\epsilon} \\
	\underline{\kappa}
	\end{Bmatrix}
\end{align}
where $\underline{\underline{\mathcal{M}}}$ and
$\underline{\underline{\mathcal{C}}}$ are the $6 \times 6$ sectional mass
and stiffness matrices,respectively, note that they are not really tensors;
$\underline{\epsilon}$ and $\underline{\kappa}$ are the 1D strains and
curvatures, respectively. $\underline{\omega}$ is the angular velocity
vector that is defined by the rotation tensor $\underline{\underline{R}}$ as
$\underline{\omega} = \mathrm{axial}(\dot{\underline{\underline{R}}}~\underline{\underline{R}})$.

For a displacement-based finite element implementation, there are six
degree-of-freedoms(DoFs) at each node: 3 displacement components and 3
rotation components. Here we use $\vec{q}$ to denote the elemental
displacement array as $\underline{q}=\left[
\underline{u}^T~~\underline{p}^T\right]$ where $\vec{u}$ is the 1D
displacement and $\vec{p}$ is the rotation parameter vector. The
acceleration array can thus be defined as $\underline{a}=\left[
\ddot{\underline{u}}^T~~ \dot{\underline{\omega}}^T \right]$. For nonlinear
finite element analysis, the discretized and incremental forms of
displacement, velocity, and acceleration array are written as
\begin{align}
	\label{Discretized}
	\underline{q} (x_1) &= \underline{\underline{N}} ~\hat{\underline{q}}~~~~\Delta \underline{q}^T = \left[ \Delta \underline{u}^T~~\Delta \underline{p}^T \right] \\
	\underline{v}(x_1) &= \underline{\underline{N}}~\hat{\underline{v}}~~~~\Delta \underline{v}^T = \left[\Delta \underline{\dot{u}}^T~~\Delta \underline{\omega}^T \right] \\
%	\label{Incremental}
	\underline{a}(x_1) &= \underline{\underline{N}}~ \hat{\underline{a}}~~~~\Delta \underline{a}^T = \left[ \Delta \ddot{\underline{u}}^T~~\Delta \dot{\underline{\omega}}^T \right]	
\end{align}
where $\tens{N}$ is the shape function matrix and $(\hat{\bullet})$ denotes
a column matrix of nodal values.  The governing equations for beams are
highly nonlinear so that a linearization process is needed. According to
Ref~\cite{Bauchau:2010}, the linearized governing equations in
Eq.~\eqref{GovernGEBT-1} and \eqref{GovernGEBT-2} are in the form of
\begin{equation}
	\label{LinearizedEqn}
	\hat{\underline{\underline{M}}} \Delta \hat{\underline{a}} +\hat{\underline{\underline{G}}} \Delta \hat{\underline{v}}+ \hat{\underline{\underline{K}}} \Delta \hat{\underline{q}} = \hat{\underline{F}}^{ext} - \hat{\underline{F}}
\end{equation} 
where the $\hat{\tens{M}}$, $\hat{\tens{G}}$, and $\hat{\tens{K}}$ are the
elemental mass, gyroscopic, and stiffness matrices, respectively;
$\hat{\vec{F}}$ and $\hat{\vec{F}}^{ext}$ are the elemental forces and
externally applied loads, respectively. They are defined as follows
\begin{align}
	\label{hatM} 
	\hat{\tens{M}}&= \int_0^l \underline{\underline{N}}^T \mathcal{\underline{\underline{M}}} ~\underline{\underline{N}} dx_1 \\
	\label{hatG}
	\hat{\tens{G}} &= \int_0^l \tens{N}^T \tens{\mathcal{G}}^I~\tens{N} dx_1\\ 
	\label{hatK}
	\hat{\tens{K}}&=\int_0^l \left[ \tens{N}^T (\tens{\mathcal{K}}^I + \mathcal{\tens{Q}})~ \tens{N} + \tens{N}^T \mathcal{\tens{P}}~ \tens{N}^\prime + \tens{N}^{\prime T} \mathcal{\tens{C}}~ \tens{N}^\prime + \tens{N}^{\prime T} \mathcal{\tens{O}}~ \tens{N} \right] d x_1 \\	
	\label{hatF}
	\hat{\vec{F}} &= \int_0^l (\tens{N}^T \vec{\mathcal{F}}^I + \tens{N}^T \mathcal{\vec{F}}^D + \tens{N}^{\prime T} \mathcal{\vec{F}}^C)dx_1 \\
	\label{hatFext}
	\hat{\vec{F}}^{ext}& = \int_0^l \tens{N}^T \mathcal{\vec{F}}^{ext} dx_1 
\end{align}
The new matrix notations in Eq.~\eqref{hatM} to \eqref{hatFext} are briefly
introduced here. $\mathcal{\tens{M}}$ is the sectional mass matrix resolved
in inertial system; $\mathcal{\vec{F}}^C$ and $\mathcal{\vec{F}}^D$ are
elastic forces obtained from Eq.~\eqref{GovernGEBT-1} and
\eqref{GovernGEBT-2} as
\begin{align}
	\label{FC}
	\mathcal{\vec{F}}^C &= \begin{Bmatrix}
         \vec{F} \\
	\vec{M}
	\end{Bmatrix} = \tens{\mathcal{C}} \begin{Bmatrix}
	\vec{\epsilon} \\
	\vec{\kappa}
	\end{Bmatrix} \\
	\label{FD}
	\mathcal{\vec{F}}^D & = \begin{bmatrix}
	\underline{\underline{0}} & \underline{\underline{0}}\\
	(\tilde{x}_0^\prime+\tilde{u}^\prime)^T & \underline{\underline{0}}
	\end{bmatrix}
	\mathcal{\vec{F}}^C \equiv \tens{\Upsilon}~ \mathcal{\vec{F}}^C
\end{align}
where $\underline{\underline{0}}$ denotes a $3 \times 3$ null matrix. The $\tens{\mathcal{G}}^I$, $\tens{\mathcal{K}}^I$,  $\mathcal{\tens{O}}$, $\mathcal{\tens{P}}$, $\mathcal{\tens{Q}}$, and $\vec{\mathcal{F}}^I$ in Eq.~\eqref{hatG}, Eq.~\eqref{hatK}, and Eq.~\eqref{hatF} are defined as
\begin{align}
        \label{mathcalG}
        \tens{\mathcal{G}}^I &= \begin{bmatrix}
        \tens{0} & (\tilde{\omega} m \vec{\eta})^T+\tilde{\omega} m \tilde{\eta}^T  \\
        \tens{0} & \tilde{\omega} \tens{\varrho}-\widetilde{\tens{\varrho} \vec{\omega}}
        \end{bmatrix} \\
        \label{mathcalK}
        \tens{\mathcal{K}}^I &= \begin{bmatrix}
        \tens{0} & \dot{\tilde{\omega}}m\tilde{\eta}^T + \tilde{\omega} \tilde{\omega}m\tilde{\eta}^T  \\
        \tens{0} & \ddot{\tilde{u}}m\tilde{\eta} + \tens{\varrho} \dot{\tilde{\omega}}-\widetilde{\tens{\varrho} \vec{\dot{\omega}}}+\tilde{\omega} \tens{\varrho} \tilde{\omega} - \tilde{\omega}  \widetilde{\tens{\varrho} \vec{\omega}}
        \end{bmatrix}\\
	\label{mathcalO}
	\mathcal{\tens{O}} &= \begin{bmatrix}
	\tens{0} & \tens{C}_{11} \tilde{E_1} - \tilde{F} \\
	\tens{0}& \tens{C}_{21} \tilde{E_1} - \tilde{M}
	\end{bmatrix} \\
	\label{mathcalP}
	\mathcal{\tens{P}} &= \begin{bmatrix}
	\tens{0} & \tens{0} \\
	\tilde{F} +  (\tens{C}_{11} \tilde{E_1})^T & (\tens{C}_{21} \tilde{E_1})^T
	\end{bmatrix}  \\
	\label{mathcalQ}
	\mathcal{\tens{Q}} &= \tens{\Upsilon}~ \mathcal{\tens{O}} \\
	\label{mathcalF}
	\vec{\mathcal{F}}^I &= \begin{Bmatrix}
	m \ddot{\vec{u}} + (\dot{\tilde{\omega}} + \tilde{\omega} \tilde{\omega})m \vec{\eta} \\
	m \tilde{\eta} \ddot{\vec{u}} +\tens{\varrho}\dot{\vec{\omega}}+\tilde{\omega}\tens{\varrho}\vec{\omega}
	\end{Bmatrix}
\end{align}
The following notations were introduced to simply the writing of the above expressions
\begin{align}
    \label{E1}
    \vec{E}_1 &= \vec{x}_0^\prime + \vec{u}^\prime \\
    \label{PartC}
    \tens{\mathcal{C}} &= \begin{bmatrix}
    \tens{C}_{11} & \tens{C}_{12} \\
    \tens{C}_{21} & \tens{C}_{22}
    \end{bmatrix}
\end{align}
 
The derivation and linearization of governing equations of geometrically exact beam theory can be found in Ref~\cite{Bauchau:2010}.

\section{Spectral Finite Element Implementation}
The displacement fields in a element can be interpolated as
\begin{align}
    \label{InterpolateDisp}
    \vec{u}(s) &= h^k(s) \vec{\hat{u}}^k \\
    \label{InterpolateDispp}
    \vec{u}^\prime(s) &= h^{k\prime}(s) \vec{\hat{u}}^k
\end{align}
where $\vec{u}$ is the displacement field, $h^k(s), k=1,2,...n$ are the
shape functions from first to the $n^{th}$ node; $\vec{\hat{u}}$ is the
nodal values of the displacement field. However, as discussed in Ref~\cite{Bauchau-etal:2009}, the three-dimensional rotation field cannot be
simply interpolated as the displacement field in the form of
\begin{align}
    \label{InterpolateRot}
    \vec{c}(s) &= h^k(s) \vec{\hat{c}}^k \\
    \label{InterpolateRotp}
    \vec{c}^\prime(s) &= h^{k \prime}(s) \vec{\hat{c}}^k 
\end{align}    
where $\vec{c}$ is the rotation field in a element and $\vec{\hat{c}}^k$ is
the nodal value at the $k^{th}$ node, for three reasons: 1) rotations do not
form a linear space so that they have to be composed instead of added; 2)
rescaling operation is needed to eliminate the singularity existing in the
vectorial rotation parameters; 3) it is lack of objectivity, which is
defined by Crisfield and Jeleni\'c\cite{Crisfield1999} refers
to the invariance of strain measures computed through interpolation to the
addition of a rigid body motion. Therefore, we adopt a more robust
interpolation approach proposed by Crisfield and Jeleni\'c
\cite{Crisfield1999} to deal with the finite rotations.
\begin{description}
    \item[Step 1:] Compute the nodal relative rotations, $\vec{\hat{r}}^k$ by removing the rigid body rotation, $\vec{\hat{c}}^1$, from the finite rotation at each node, $\vec{\hat{r}}^k = \vec{\hat{c}}^{1-} \oplus \vec{\hat{c}}^k$.
    \item[Step 2:] Interpolate the relative rotation field: $\vec{r}(s) = h^k(s) \vec{\hat{r}}^k$ and $\vec{r}^\prime(s) = h^{k \prime}(s) \vec{\hat{r}}^k$. Find the curvature field $\vec{\kappa}(s) = \tens{R}(\vec{\hat{c}}^1) \tens{H}(\vec{r}) \vec{r}^\prime$.
    \item[Step 3:] Restore the rigid body rotation removed in Step 1: $\vec{c}(s) = \vec{\hat{c}}^1 \oplus \vec{r}(s)$.
\end{description} 
where $H$ is the tangent tensor that relates the curvature vector $k$ and rotation vector $p$ as
\begin{equation}
    \label{Tensor}
    \vec{k} = \tens{H}~ \vec{p}^\prime
\end{equation}
In the LSFE approach, shape functions (e.g., those composing $\tens{N}$) are
$n^{th}$-order Lagrangian interpolants, where nodes are located at the $n+1$
GLL-quadrature points in the $[-1,1]$ element natural-coordinate domain.
\textcolor{red}{Need more work here: a figure shows some LS elements (non-evenly placed internal nodes) and a short discussion of its advantages.}
The implementation of GEBT into the FE code, BeamDyn, carries out the integration in the space domain by using reduced Gauss quadrature. Time integration is performed using generalized-$\alpha$ scheme, which is a unconditionally stable, second-order accurate algorithm. More details regarding on the generalized-$\alpha$ method can be found in Ref~\cite{Chung-Hulbert:1993,Bauchau:2010}


\section{Numerical Examples}

\subsection{Example 1: Static bending of a cantilever beam}

The first example is a benchmark problem for geometrically nonlinear
analysis of beams \cite{Simo1985,Xiao-Zhong:2012}. We calculate the static
deflection of a cantilever beam that is subjected at its free end to
a constant moment $M$.  The length of the beam $L$ is $10\,in$ and the cross-sectional stiffness 
matrix is given below:
\begin{equation}
    \label{StifE1}
    C^* = 10^3 \times \begin{bmatrix}
	1770 & 0    & 0    & 0    & 0    & 0   \\
	     & 1770 & 0    & 0    & 0    & 0   \\
	     &      & 1770 & 0    & 0    & 0   \\
	     &      &      & 8.16 & 0    & 0   \\
	     &      &      &      & 86.9 & 0   \\
	     &      &      &      &      & 215
\end{bmatrix}
\end{equation}
It is pointed out that the term with a asterisk denotes it is resolved in the material coordinate system. A sketch of this case can be found in Figure~\ref{E1Sketch}.
\begin{figure}
    \centering
    \includegraphics[width=5.0in]{\directory E1Sketch.eps}
    \caption{Sketch of a cantilever beam.}
    \label{E1Sketch}
\end{figure} 
The load applied at the tip is given by the following equation:
\begin{equation}
    \label{E1Load}
    M_2 = \lambda \bar{M}_2
\end{equation}
where $\bar{M}_2 = \pi \frac{EI_2}{L}$. The parameter $\lambda$ will vary between $0$ and $2$. In 
this case, the beam is discretized with two $5^{th}$ order elements. The deformations of the beam are shown in Figure~\ref{E1Deform}.
\begin{figure}
    \centering
    \includegraphics[width=5.0in]{\directory E1Deform.eps}
    \caption{Deformations of a cantilever beam under several constant bending moments.}
    \label{E1Deform}
\end{figure}
The calculated results are compared with the analytical solution, which can be found in Ref.\cite{Mayo-etal:2004} as
\begin{equation}
    \label{E1Analytical}
    u_1 = \rho sin \left( \frac{x_1}{\rho} \right) - x_1~~~~~u_3 = \rho \left(1-cos\left(\frac{x_1}{\rho}\right) \right)
\end{equation}
The results can be found in Table~\ref{E1u1} and \ref{E1u3}, respectively. Good agreement can be observed between these two sets of results.
\begin{table}[tbp]
\centering 
\caption{Axial displacement $u_1$ of a cantilever beam subject to a constant moment (in inches).}
\label{E1u1} 
	\begin{tabular}{| l | l | l | l |}
    	\hline
    	$\lambda$ & Analytical & BeamDyn  & \% Error \\ \hline
    	0.4       & -2.4317    & -2.4317  & 0.00       \\ \hline
    	0.8       & -7.6613    & -7.6613  & 0.00       \\ \hline
    	1.2       & -11.5591   & -11.5591 & 0.00       \\ \hline
    	1.6       & -11.8921   & -11.8921 & 0.00       \\ \hline
    	2.0       & -10.0000   & -10.0000  & 0.00       \\ \hline
    \end{tabular}
\end{table}

\begin{table}[tbp]
\centering 
\caption{Vertical displacement $u_3$ of a cantilever beam subject to a constant moment (in inches).}
\label{E1u3} 
	\begin{tabular}{| l | l | l | l |}
    	\hline
    	$\lambda$ & Analytical & BeamDyn & \% Error \\ \hline
    	0.4       & 5.4987     & 5.4987  & 0.00   \\ \hline
    	0.8       & 7.1978     & 7.1979  & 0.0013   \\ \hline
    	1.2       & 4.7986     & 4.7986  & 0.00   \\ \hline
    	1.6       & 1.3747     & 1.3747  & 0.00   \\ \hline
    	2.0       & 0.0000     & 0.0000  & 0.00   \\ \hline
    \end{tabular}
 \end{table}
 The rotation parameters at each node along beam axis $x_1$ obtained from BeamDyn are plotted in Figure~\ref{E1Rot} for $\lambda = 0.8$ and $\lambda = 2.0$, respectively. It is noted that the three-dimensional rotations are represented by Wiener-Milenkovi\'c parameter defined in the following equation:
 \begin{equation}
     \vec{p} = 4 tan\frac{\phi}{4} \bar{n}
     \label{WMParameter}
 \end{equation}
 where $\phi$ is the rotation angle and $\bar{n}$ is the unit vector of rotation axis. The singularity exists in the above definition can be removed by a rescaling operation, which can be observed in Figure~\ref{E1Rot}.
\begin{figure}
    \centering
    \includegraphics[width=5.0in]{\directory E1Rot.eps}
    \caption{Wiener-Milenkovi\'c rotation parameters along beam axis $x_1$ .}
    \label{E1Rot}
\end{figure}
Figure~\ref{E1Conv} shows the normalized error $\epsilon(u)$, where $u$ is the tip displacement (at $x=L$), as a function of the number of model nodes for the calculation with Dymore quadratic elements (QE) and a single Legendre spectral element (LSE), where
\begin{equation}
    \label{E1Error}
    \epsilon(u) = \left| \frac{u-u^a}{u^a} \right|
\end{equation}
and $u$ is the test solution and $u^a$ is the analytical solution. The parameter $\lambda$ is set to $1.0$ for this case. The Legendre spectral elements (with $p$-refinement) exhibit highly desirable exponential convergence to machine precision error.
\begin{figure}
    \centering
    \begin{tabular}{c}
    \subfloat[$u_1$]{\label{E1Conv:u1}\includegraphics[width=3.0 in]{\directory  E1Convu1.eps}} \qquad
\subfloat[$u_3$]{\label{E1Conv:u3}\includegraphics[width=3.0in]{\directory  E1Convu3.eps}}\\
\end{tabular}
\caption{Normalized error of the (a) $u_1$ and (b) $u_3$ displacements as a function of the total number of nodes}
\label{E1Conv}
\end{figure}

\subsection{Example 2: Static analysis of a composite beam}
The second example is to show the capability of BeamDyn for composite beams with elastic couplings. The cantilever beam used in this case is $10$ inches long with a boxed cross-section made of composite materials that can be found in Ref.\cite{Yu-etal:2002}. Readers are referred to Figure~\ref{E1Sketch} for sketch of this example. The stiffness matrix is given as
\begin{equation}
C^* = 10^3 \times \begin{bmatrix}
	1368.17 & 0     & 0     & 0      & 0      & 0      \\
	0       & 88.56 & 0     & 0      & 0      & 0      \\
	0       & 0     & 38.78 & 0      & 0      & 0      \\
	0       & 0     & 0     & 16.96  & 17.61  & -0.351 \\
	0       & 0     & 0     & 17.61  & 59.12  & -0.370 \\
	0       & 0     & 0     & -0.351 & -0.370 & 141.47
\end{bmatrix}
\label{E2Stif}
\end{equation}
A concentrated force $P = 150~lbs$ along $x_3$ direction is applied at the free tip. In BeamDyn analysis, the beam is meshed with two $5^{th}$ order elements. The displacements and rotation parameters at each node along beam axis are plotted in Figure~\ref{E2U}.  It is noted that the coupling effects exist between twist and two bendings. The applied in-plane force leads to a fairly large twist angle due to the bending-twist coupling, which can be observed in Figure.~\ref{E2Rot}. It is also noted that the internal nodes of Legendre Spectral Finite Elements are not evenly placed, which is different from conventional elements. 

\begin{figure}
    \centering
    \begin{tabular}{c}
    \subfloat[$Displacements$]{\label{E2Disp}\includegraphics[width=3.0 in]{\directory  E2Disp.eps}} \qquad
\subfloat[$Rotations$]{\label{E2Rot}\includegraphics[width=3.0in]{\directory  E2Rot.eps}}\\
\end{tabular}
\caption{Displacements and rotation parameters along beam axis for Example 2.}
\label{E2U}
\end{figure}

The tip displacements and rotations are compared with those obtained by Dymore in Table~\ref{E2Tip} for verification, where the beam is meshed with 10 $3^{rd}$ order elements. Good agreement can be observed between BeamDyn and Dymore results.
\begin{table}[tbp]
\centering 
\caption{Tip displacements and rotation parameters of a composite beam in Example 2}
\label{E2Tip} 
	\begin{tabular}{| l | l | l | l | l | l | l |}
    	\hline
    	 & $u_1$ & $u_2$ & $u_3$  & $p_1$ & $p_2$ & $p_3$  \\ 
	 \hline
	 BeamDyn & -0.09064 & -0.06484 & 1.22998 & 0.18445 & -0.17985 & 0.00488 \\
	 \hline
	 Dymore & -0.09064 & -0.06483 & 1.22999 & 0.18443 & -0.17985 & 0.00488 \\
    	\hline
    \end{tabular}
\end{table}

\subsection{Example 3: dynamic analysis of a composite beam under sinusoidal force at the tip}

The last example is a transient analysis of a composite beam with boxed cross-section that is used in Example 2. The beam has the same geometry and boundary conditions as the one in previous example. The mass sectional properties are given by VABS \cite{Yu-etal:2002,Wang-Yu:2012} as
\begin{equation}
    \label{E3Mass}
    M^* = 10^{-2} \times \begin{bmatrix}
	8.538 & 0     & 0     & 0      & 0      & 0      \\
	0       & 8.538 & 0     & 0      & 0      & 0      \\
	0       & 0     & 8.538 & 0      & 0      & 0      \\
	0       & 0     & 0     & 1.4433  & 0  & 01 \\
	0       & 0     & 0     & 0  & 0.40972  &0 \\
	0       & 0     & 0     & 0 & 0 & 1.0336
\end{bmatrix}    
\end{equation}
The beam is divided into two $5^{th}$ order elements in the current calculation and a sinusoidal point force is applied at the free tip in the $x_3$ direction given as
\begin{equation}
    \label{E3AppliedForce}
    P = A_F~sin(\omega_F~t)
\end{equation}
where $A_F = 1.0 \times 10^2~lbs$ and $\omega_F = 10~rad/s$ (see Figure~\ref{E3SinForce}). The time step used in this example is $0.005~s$ so that a set of converged results can be achieved. The tip displacement and rotation histories of the beam are plotted in Figure~\ref{E3U}. Note that all the components, including three displacements and three rotations, are non-zero due to the elastic coupling effects. The time histories of the stress resultants at the root of the beam are given in Figure~\ref{E3F}.
\begin{figure}
    \centering
    \includegraphics[width=5.0in]{\directory E3SinForce.eps}
    \caption{The sinusoidal vertical force in Example 3 .}
    \label{E3SinForce}
\end{figure}

\begin{figure}
    \centering
    \begin{tabular}{c}
    \subfloat[$u_1$]{\label{E3U:u1}\includegraphics[width=3.0 in]{\directory  E3Tipu1.eps}} \qquad
\subfloat[$u_2$]{\label{E3U:u2}\includegraphics[width=3.0in]{\directory  E3Tipu2.eps}} \\
\subfloat[$u_3$]{\label{E3U:u3}\includegraphics[width=3.0 in]{\directory  E3Tipu3.eps}} \qquad
\subfloat[$p_1$]{\label{E3U:p1}\includegraphics[width=3.0 in]{\directory  E3Tipp1.eps}} \\
\subfloat[$p_2$]{\label{E3U:p2}\includegraphics[width=3.0 in]{\directory  E3Tipp2.eps}} \qquad
\subfloat[$p_3$]{\label{E3U:p3}\includegraphics[width=3.0 in]{\directory  E3Tipp3.eps}} \\
\end{tabular}
\caption{Tip displacement and rotation histories of a composite beam under vertical load.}
\label{E3U}
\end{figure} 

\begin{figure}
    \centering
    \begin{tabular}{c}
    \subfloat[$F_1$]{\label{E3F:F1}\includegraphics[width=3.0 in]{\directory  E3RootF1.eps}} \qquad
\subfloat[$F_2$]{\label{E3F:F2}\includegraphics[width=3.0in]{\directory  E3RootF2.eps}} \\
\subfloat[$F_3$]{\label{E3F:F3}\includegraphics[width=3.0 in]{\directory E3RootF3.eps}} \qquad
\subfloat[$M_1$]{\label{E3F:M1}\includegraphics[width=3.0 in]{\directory  E3RootM1.eps}} \\
\subfloat[$M_2$]{\label{E3F:M2}\includegraphics[width=3.0 in]{\directory  E3RootM2.eps}} \qquad
\subfloat[$M_3$]{\label{E3F:M3}\includegraphics[width=3.0 in]{\directory  E3RootM3.eps}} \\
\end{tabular}
\caption{Stress resulant time histories at the root of a composite beam.}
\label{E3F}
\end{figure} 
 
\section{Conclusion}
This paper presents a displacement-based implementation of geometrically exact beam theory. The Legendre spectral finite elements are adopted to discretize the beam in the space domain. Numerical examples were presented that demonstrate the capability of BeamDyn, a beam solver for wind turbine analysis developed by NREL. A benchmark static problem for nonlinear beam was studied first. The agreement between the results calculated by BeamDyn and analytical solution are excellent. Moreover, a convergence study has been conducted where the convergence rate of Legendre spectral elements are compared with the conventional $2^{nd}$ order elements. Exponential convergence rates were observed as expected for this type of element. A composite cantilever beam were studied both statically and dynamically. The static results are verified against those obtained by Dymore. The elastic coupling effects were shown in these two cases. It concludes that BeamDyn is a powerful tool for composite beam analysis that can be used as a module in the FAST modularization framework.

\section*{Acknowledgments} 

This work was supported by the U.S. Department of Energy under Contract No.\
DE-AC36-08-GO28308 with the National Renewable Energy Laboratory. Support
was provided through a Laboratory Directed Research and Development grant
\textit{High-Fidelity Computational Modeling of Wind-Turbine Structural
Dynamics}.  The authors acknowledge Professor Oliver A.  Bauchau for the technical
discussions on the 3D rotation parameters.
 

\bibliographystyle{aiaa}
\bibliography{references,local_bib}

\end{document}
