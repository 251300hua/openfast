%&LaTeX
\section{Introduction} 

%Wind power is becoming one of the most important renewable energy sources in
%the United States as demonstrated by the fact that the electricity produced
%from wind amounted to 3.56\% of all generated electrical energy for the 12
%months until March 2013\cite{WindWiki}. 
Wind power is becoming one of the most important renewable-energy sources in
the United States.
%as demonstrated by the fact that the electricity produced
%from wind amounted to 3.56\% of all generated electrical energy for the 12
%months until March 2013\cite{WindWiki}. 
In recent years, the size of wind
turbines has been increasing immensely to lower the cost, which, because of
weight restrictions, also leads
to highly flexible turbine blades. This huge electro-mechanical system poses
a significant challenge for engineering design and analysis. Although
possible with modern super computers, direct three-dimensional (3D)
structural analysis is so computationally expensive that engineers are
always seeking for efficient high-fidelity simplified models.

Beam models are widely used to represent and analyze engineering structures
that have one of its dimensions much larger than the other two. Many
engineering components can be idealized as beams: bridges in civil
engineering, joists and lever arms in heavy-machine industries, and
helicopter rotor blades. The blades, tower, and shaft in a wind turbine
system can be considered as beams. In the weight-critical applications of
beam structures, like high-aspect-ratio wings in aerospace and wind energy,
composite materials are attractive due to their superior strength-to-weight
and stiffness-to-weight ratios.  However, analysis of composite-materials
structures is more difficult than their isotropic counterparts due to
elastic-coupling effects. The geometrically exact beam theory (GEBT) first
first proposed by Reissner\cite{Ressiner1973}, is a method that
has proven powerful for analysis of highly flexible composite beams in the
helicopter engineering community. During the past several decades, much
effort has been invested in this area. Simo\cite{Simo1985} and Simo and
Vu-Quoc\cite{Simo1986} extended Reissner's work to deal with
three-dimensional (3D) dynamic problems. Jeleni\'c and
Crisfield\cite{Crisfield1999} implemented this theory using the finite-element method where a new approach for interpolating the rotation field was
proposed that preserves the geometric exactness. Betsch and
Steinmann\cite{Betsch2002} circumvented the interpolation of rotation by
introducing a re-parameterization of the weak form corresponding to the
equations of motion of GEBT. It is noted that Ibrahimbegovi\'c and his
colleagues implemented this theory for static\cite{Ibrahim1995} and
dynamic\cite{Ibrahim1998} analysis. In contrast to the displacement-based
implementations, the geometric exact beam theory has also been formulated by
mixed finite elements where both the primary and dual field are
independently interpolated\cite{CookFEM}. In the mixed formulation, all of
the necessary ingredients, including Hamilton's principle and kinematic
equations, are combined in a single variational formulation statement;
Lagrange multipliers, motion variables, generalized strains, forces and
moments, linear and angular momenta, and displacement and rotation variables
are considered as independent quantities. Yu et al.\cite{YuGEBT,
Wang:GEBT2013} presented the implementation of GEBT in a mixed formulation;
various rotation parameters were investigated and the code was validated
against analytical and numerical solutions. Readers are referred to
Hodges\cite{HodgesBeamBook}, where comprehensive derivations and
discussions on nonlinear composite-beam theories can be found.

Legendre spectral finite elements\cite{Patera:1984,Ronquist:1987} (LSFEs) are
$p$-type finite elements whose shape functions are Lagrangian interpolants
with node locations at the Gauss-Lobatto-Legendre (GLL) points. LSFEs combine
the accuracy of global spectral methods with geometric flexibility of {\it
h}-type FEs. The spectral FEs have seen successful use in
the simulation of fluid dynamics\cite{Ronquist:1987, Patera:1984,
Deville:2002}, two-dimensional elastic wave propagation in solid media in
geophysics \cite{Komatitsch:1998}, elastodynamics \cite{Sridhar:2006}, and
acoustic wave propagation \cite{Sprague:2004}. However, it has seen limited
application to dynamic analysis of
beam\cite{Ben-Tal-etal:1995,Ben-Tal-etal:1996,Kudela-etal:2007a,Sprague-Geers:2008}
and plate elements\cite{Zrahia-Bar-Yoseph:1995,
Kudela-etal:2007b,Sprague-Brito:2012}.   \mas{we need to add references on
"quadrature elements"}

In this paper, we present a displacement-based implementation of
geometrically exact beam theory using LSFEs.  This work builds on a previous
effort which showed the implementation of three-dimensional rotation
parameters\cite{Wang:GEBT2013} and a demonstration example of
two-dimensional nonlinear spectral beam elements\cite{Wang:SFE2013} for
static deformation.  The code implemented in this work is in accordance to
FAST Modularization Framework \cite{Jonkman:2013},  which allows simulation
of a whole turbine under realistic operating conditions.
\mas{EXPAND ON FAST MODULARIZATION}

The paper is organized as follows.  The theoretical foundation of the
geometrically exact beam theory is introduced first. Then the GEBT
discretization by LSFEs is discussed. Finally, verification examples are
provided to show the accuracy and efficiency of the GEBT LSFEs  for
isotropic  and composite beams.  

