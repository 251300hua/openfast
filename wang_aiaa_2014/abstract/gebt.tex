%&LaTeX
\section{Geometrically Exact Beam Theory}

For completeness, this section reviews the geometrically exact beam theory and
linearization process of the governing equations. The content of this
section can be found in many other papers and textbooks.
Figure~\ref{Kinematics} shows a beam in its initial undeformed and deformed
states. A reference frame $\mathbf{b}_i$, for $i=\{ 1,2,3 \}$, is introduced along the beam axis
for the undeformed state; a frame $\mathbf{B}_i$ is introduced along each
point of the deformed beam axis. Curvilinear-coordinate $x_1$ defines the
intrinsic parameterization of the reference line; \textcolor{red}{ and similarly, $s$ denotes the deformed reference line.}

\begin{figure}
\centering
\includegraphics[width=5.0in]{\directory Kinematics.eps}
\caption{Schematic of beam deformation} \label{Kinematics}
\end{figure}

In this paper, we use matrix notation to denote vectorial or vectorial-like
quantities. For example, we use a underline to denote a vector
$\underline{u}$, a bar to denote unit vector $\bar{n}$, and double underline
to denote a tensor $\underline{\underline{\Delta}}$. Note that sometimes the
underlines only denote the dimension of the corresponding matrix. The
governing equations of motion for geometrically exact beam theory can be written
as\cite{Bauchau:2010}
\begin{align}
	\label{GovernGEBT-1}
	\dot{\underline{h}} - \underline{F}^\prime &= \underline{f} \\
	\label{GovernGEBT-2}
	\dot{\underline{g}} + \dot{\tilde{u}} \underline{h} - \underline{M}^\prime - (\tilde{x}_0^\prime + \tilde{u}^\prime) \underline{F} &= \underline{m}
\end{align}
where $\vec{h}$ and $\vec{g}$ are the linear and angular momenta resolved in
the inertial coordinate system, respectively; $\vec{F}$ and $\vec{M}$ are
the beam's sectional forces and moments, respectively; $\vec{u}$ is the 1D
displacement of the reference line; $\vec{x}_0$ is the initial position vector of a
point along the beam's reference line; $\vec{f}$ and $\vec{m}$ are the
distributed force and moment applied to the beam structure.  
A prime 
indicates a derivative with respect to the beam axis
$x_1$ and an overdot indicates a derivative with respect to time. The
tilde operator,  i.e., $(\skew{\cdot})$, denotes a second-order, skew-symmetric
tensor corresponding to the given vector. In the literature, it is also
termed as "cross-product matrix". For example, for the vector
$\overline{n}$,
\[
	\skew{n} = 
	     		\begin{bmatrix}
			0 & -n_3 & n_2 \\
			n_3 & 0 & -n_1 \\
			-n_2 & n_1 & 0\\
			\end{bmatrix}	
\]
The constitutive equations relate the velocities to the momenta and the one-dimensional strain measures to the sectional resultants as
\begin{align}
	\label{ConstitutiveMass}
	\begin{Bmatrix}
	\underline{h} \\
	\underline{g}
	\end{Bmatrix}
	= \underline{\underline{\mathcal{M}}} \begin{Bmatrix}
	\dot{\underline{u}} \\
	\underline{\omega}
	\end{Bmatrix} \\
	\label{ConstitutiveStiff}
	\begin{Bmatrix}
	\underline{F} \\
	\underline{M}
	\end{Bmatrix}
	= \underline{\underline{\mathcal{C}}} \begin{Bmatrix}
	\underline{\epsilon} \\
	\underline{\kappa}
	\end{Bmatrix}
\end{align}
where $\underline{\underline{\mathcal{M}}}$ and
$\underline{\underline{\mathcal{C}}}$ are the $6 \times 6$ sectional mass
and stiffness matrices,respectively, note that they are not really tensors;
$\underline{\epsilon}$ and $\underline{\kappa}$ are the 1D strains and
curvatures, respectively. $\underline{\omega}$ is the angular velocity
vector that is defined by the rotation tensor $\underline{\underline{R}}$ as
$\underline{\omega} = \mathrm{axial}(\dot{\underline{\underline{R}}}~\underline{\underline{R}})$.

For a displacement-based finite element implementation, there are six
degree-of-freedoms (DoFs) at each node: 3 displacement components and 3
rotation components. Here we use $\vec{q}$ to denote the elemental
displacement array as $\underline{q}=\left[
\underline{u}^T~~\underline{p}^T\right]$ where $\vec{u}$ is the 1D
displacement and $\vec{p}$ is the rotation parameter vector. The
acceleration array can thus be defined as $\underline{a}=\left[
\ddot{\underline{u}}^T~~ \dot{\underline{\omega}}^T \right]$. For nonlinear
finite element analysis, the discretized and incremental forms of
displacement, velocity, and acceleration array are written as
\begin{align}
	\label{Discretized}
	\underline{q} (x_1) &= \underline{\underline{N}} ~\hat{\underline{q}}~~~~\Delta \underline{q}^T = \left[ \Delta \underline{u}^T~~\Delta \underline{p}^T \right] \\
	\underline{v}(x_1) &= \underline{\underline{N}}~\hat{\underline{v}}~~~~\Delta \underline{v}^T = \left[\Delta \underline{\dot{u}}^T~~\Delta \underline{\omega}^T \right] \\
%	\label{Incremental}
	\underline{a}(x_1) &= \underline{\underline{N}}~ \hat{\underline{a}}~~~~\Delta \underline{a}^T = \left[ \Delta \ddot{\underline{u}}^T~~\Delta \dot{\underline{\omega}}^T \right]	
\end{align}
where $\tens{N}$ is the shape function matrix and $(\hat{\bullet})$ denotes
a column matrix of nodal values. \textcolor{red}{It is noted that given the ``untensorial'' nature, we need to adopt some special algorithm to deal with the 3D rotations which will be introduced in the next section.}   The governing equations for beams are
highly nonlinear so that a linearization process is needed. According to
Bauchau\cite{Bauchau:2010}, the linearized governing equations in
Eq.~\eqref{GovernGEBT-1} and \eqref{GovernGEBT-2} are in the form of
\begin{equation}
	\label{LinearizedEqn}
	\hat{\underline{\underline{M}}} \Delta \hat{\underline{a}} +\hat{\underline{\underline{G}}} \Delta \hat{\underline{v}}+ \hat{\underline{\underline{K}}} \Delta \hat{\underline{q}} = \hat{\underline{F}}^{ext} - \hat{\underline{F}}
\end{equation} 
where the $\hat{\tens{M}}$, $\hat{\tens{G}}$, and $\hat{\tens{K}}$ are the
elemental mass, gyroscopic, and stiffness matrices, respectively;
$\hat{\vec{F}}$ and $\hat{\vec{F}}^{ext}$ are the elemental forces and
externally applied loads, respectively. They are defined as follows
\begin{align}
	\label{hatM} 
	\hat{\tens{M}}&= \int_0^l \underline{\underline{N}}^T \mathcal{\underline{\underline{M}}} ~\underline{\underline{N}} dx_1 \\
	\label{hatG}
	\hat{\tens{G}} &= \int_0^l \tens{N}^T \tens{\mathcal{G}}^I~\tens{N} dx_1\\ 
	\label{hatK}
	\hat{\tens{K}}&=\int_0^l \left[ \tens{N}^T (\tens{\mathcal{K}}^I + \mathcal{\tens{Q}})~ \tens{N} + \tens{N}^T \mathcal{\tens{P}}~ \tens{N}^\prime + \tens{N}^{\prime T} \mathcal{\tens{C}}~ \tens{N}^\prime + \tens{N}^{\prime T} \mathcal{\tens{O}}~ \tens{N} \right] d x_1 \\	
	\label{hatF}
	\hat{\vec{F}} &= \int_0^l (\tens{N}^T \vec{\mathcal{F}}^I + \tens{N}^T \mathcal{\vec{F}}^D + \tens{N}^{\prime T} \mathcal{\vec{F}}^C)dx_1 \\
	\label{hatFext}
	\hat{\vec{F}}^{ext}& = \int_0^l \tens{N}^T \mathcal{\vec{F}}^{ext} dx_1 
\end{align}
The new matrix notations in Eq.~\eqref{hatM} to \eqref{hatFext} are briefly
introduced here. $\mathcal{\tens{M}}$ is the sectional mass matrix resolved
in inertial system; $\mathcal{\vec{F}}^C$ and $\mathcal{\vec{F}}^D$ are
elastic forces obtained from Eq.~\eqref{GovernGEBT-1} and
\eqref{GovernGEBT-2} as
\begin{align}
	\label{FC}
	\mathcal{\vec{F}}^C &= \begin{Bmatrix}
         \vec{F} \\
	\vec{M}
	\end{Bmatrix} = \tens{\mathcal{C}} \begin{Bmatrix}
	\vec{\epsilon} \\
	\vec{\kappa}
	\end{Bmatrix} \\
	\label{FD}
	\mathcal{\vec{F}}^D & = \begin{bmatrix}
	\underline{\underline{0}} & \underline{\underline{0}}\\
	(\tilde{x}_0^\prime+\tilde{u}^\prime)^T & \underline{\underline{0}}
	\end{bmatrix}
	\mathcal{\vec{F}}^C \equiv \tens{\Upsilon}~ \mathcal{\vec{F}}^C
\end{align}
where $\underline{\underline{0}}$ denotes a $3 \times 3$ null matrix. The $\tens{\mathcal{G}}^I$, $\tens{\mathcal{K}}^I$,  $\mathcal{\tens{O}}$, $\mathcal{\tens{P}}$, $\mathcal{\tens{Q}}$, and $\vec{\mathcal{F}}^I$ in Eq.~\eqref{hatG}, Eq.~\eqref{hatK}, and Eq.~\eqref{hatF} are defined as
\begin{align}
        \label{mathcalG}
        \tens{\mathcal{G}}^I &= \begin{bmatrix}
        \tens{0} & (\tilde{\omega} m \vec{\eta})^T+\tilde{\omega} m \tilde{\eta}^T  \\
        \tens{0} & \tilde{\omega} \tens{\varrho}-\widetilde{\tens{\varrho} \vec{\omega}}
        \end{bmatrix} \\
        \label{mathcalK}
        \tens{\mathcal{K}}^I &= \begin{bmatrix}
        \tens{0} & \dot{\tilde{\omega}}m\tilde{\eta}^T + \tilde{\omega} \tilde{\omega}m\tilde{\eta}^T  \\
        \tens{0} & \ddot{\tilde{u}}m\tilde{\eta} + \tens{\varrho} \dot{\tilde{\omega}}-\widetilde{\tens{\varrho} \vec{\dot{\omega}}}+\tilde{\omega} \tens{\varrho} \tilde{\omega} - \tilde{\omega}  \widetilde{\tens{\varrho} \vec{\omega}}
        \end{bmatrix}\\
	\label{mathcalO}
	\mathcal{\tens{O}} &= \begin{bmatrix}
	\tens{0} & \tens{C}_{11} \tilde{E_1} - \tilde{F} \\
	\tens{0}& \tens{C}_{21} \tilde{E_1} - \tilde{M}
	\end{bmatrix} \\
	\label{mathcalP}
	\mathcal{\tens{P}} &= \begin{bmatrix}
	\tens{0} & \tens{0} \\
	\tilde{F} +  (\tens{C}_{11} \tilde{E_1})^T & (\tens{C}_{21} \tilde{E_1})^T
	\end{bmatrix}  \\
	\label{mathcalQ}
	\mathcal{\tens{Q}} &= \tens{\Upsilon}~ \mathcal{\tens{O}} \\
	\label{mathcalF}
	\vec{\mathcal{F}}^I &= \begin{Bmatrix}
	m \ddot{\vec{u}} + (\dot{\tilde{\omega}} + \tilde{\omega} \tilde{\omega})m \vec{\eta} \\
	m \tilde{\eta} \ddot{\vec{u}} +\tens{\varrho}\dot{\vec{\omega}}+\tilde{\omega}\tens{\varrho}\vec{\omega}
	\end{Bmatrix}
\end{align}
where the following notations were introduced to simplify the writing of the above expressions
\begin{align}
    \label{E1}
    \vec{E}_1 &= \vec{x}_0^\prime + \vec{u}^\prime \\
    \label{PartC}
    \tens{\mathcal{C}} &= \begin{bmatrix}
    \tens{C}_{11} & \tens{C}_{12} \\
    \tens{C}_{21} & \tens{C}_{22}
    \end{bmatrix}
\end{align} 
The derivation and linearization of governing equations of geometrically
exact beam theory can be found in Bauchau\cite{Bauchau:2010}.

It is pointed out that the three-dimensional rotations are represented by Wiener-Milenkovi\'c parameters \cite{Bauchau-etal:2008,Wang:GEBT2013} defined in the following equation:
 \begin{equation}
     \vec{p} = 4 \tan\left(\frac{\phi}{4} \right) \bar{n} 
     \label{WMParameter}
 \end{equation}
where $\phi$ is the rotation angle and $\bar{n}$ is the unit vector of
rotation axis. It can be observed that the valid range for this parameter is $|\phi| < 2 \pi$ where a singularity point will be reached at $2\pi$. The singularities existing at multiples of $\pm 2 \pi$ in the
above definition can be removed by a rescaling operation, as given in Ref~\cite{Bauchau-etal:2008}:
\begin{equation}
    \label{RescaledWM}
    \vec{r} = \begin{cases}
    4(q_0\vec{p} + p_0 \vec{q} + \tilde{p} \vec{q} ) / (\Delta_1 + \Delta_2), & \text{if } \Delta_2 \geq 0 \\
    -4(q_0\vec{p} + p_0 \vec{q} + \tilde{p} \vec{q} ) / (\Delta_1 - \Delta_2), & \text{if } \Delta_2 < 0
    \end{cases}
\end{equation}
where $\vec{p}$, $\vec{q}$, and $\vec{r}$ are the vectorial parameterization of three finite rotations such that $\tens{R}(\vec{r}) = \tens{R}(\vec{p}) \tens{R}(\vec{q})$; $p_0 = 2 - \vec{p}^T \vec{p}/8$, $q_0 = 2 - \vec{q}^T \vec{q}/8$, $\Delta_1 = (4-p_0)(4-q_0)$, and $\Delta_2 = p_0 q_0 - \vec{p}^T \vec{q}$.

