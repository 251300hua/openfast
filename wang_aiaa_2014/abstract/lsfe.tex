%&LaTeX
\section{Numerical Implementation with Legendre Spectral Finite Elements}
The displacement fields in an element are approximated as
\begin{align}
    \label{InterpolateDisp}
    \vec{u}(\xi) &= h^k(\xi) \vec{\hat{u}}^k \\
    \label{InterpolateDispp}
    \vec{u}^\prime(\xi) &= h^{k\prime}(\xi) \vec{\hat{u}}^k
\end{align}
where $h^k(\xi)$ is the $p^{th}$-order polynomial
Lagrangian-interpolant shape function of node $k$, $k=\{1,2,...,p+1]\}$, 
$\vec{\hat{u}}^k$ is
the $k^{th}$ nodal value, and $\xi \in \left[-1,1\right]$ is the element
natural coordinate.
However, as discussed in Bauchau et al.\cite{Bauchau-etal:2009}, the 
3D rotation field cannot be
simply interpolated as the displacement field in the form of
\begin{align}
    \label{InterpolateRot}
    \vec{c}(\xi) &= h^k(\xi) \vec{\hat{c}}^k \\
    \label{InterpolateRotp}
    \vec{c}^\prime(\xi) &= h^{k \prime}(\xi) \vec{\hat{c}}^k 
\end{align}    
where $\vec{c}$ is the rotation field in an element and $\vec{\hat{c}}^k$ is
the nodal value at the $k^{th}$ node, for three reasons: 1) rotations do not
form a linear space so that they must be  ``composed'' rather than added; 2)
a rescaling operation is needed to eliminate the singularity existing in the
vectorial rotation parameters; 3) the rotation field lacks objectivity,
which, as
defined by Jeleni\'c and Crisfield\cite{Crisfield1999}, refers to the
invariance of strain measures computed through interpolation to the addition
of a rigid-body motion. Therefore, we adopt the more robust interpolation
approach proposed by Jeleni\'c and Crisfield\cite{Crisfield1999} to deal
with the finite rotations. Our approach is described as follows
\begin{description}

    \item[Step 1:] Compute the nodal relative rotations, $\vec{\hat{r}}^k$,
by removing the reference rotation, $\vec{\hat{c}}^1$, from the finite
rotation at each node, $\vec{\hat{r}}^k = \vec{\hat{c}}^{1-} \oplus
\vec{\hat{c}}^k$.  \mas{Qi, what is that "1minus" on the c??}

    \item[Step 2:] Interpolate the relative-rotation field: $\vec{r}(\xi) = h^k(\xi) \vec{\hat{r}}^k$ and $\vec{r}^\prime(\xi) = h^{k \prime}(\xi) \vec{\hat{r}}^k$. Find the curvature field $\vec{\kappa}(\xi) = \tens{R}(\vec{\hat{c}}^1) \tens{H}(\vec{r}) \vec{r}^\prime$.

    \item[Step 3:] Restore the rigid-body rotation removed in Step 1: $\vec{c}(\xi) = \vec{\hat{c}}^1 \oplus \vec{r}(\xi)$.
\end{description} 
where $\tens{H}$ is the tangent tensor that relates the curvature vector $\vec{k}$ and rotation vector $\vec{p}$ as
\begin{equation}
    \label{Tensor}
    \vec{k} = \tens{H}~ \vec{p}^\prime
\end{equation}
\mas{Qi, should that k be a kappa as in Step 2?}

Note that the relative-rotation field can be computed with respect to any of
the nodes of the element; we choose node 1 as the reference node for
convenience. In the LSFE approach, shape functions (i.e., those composing $\tens{N}$) are
$p^{th}$-order Lagrangian interpolants, where nodes are located at the $p+1$
GLL-quadrature points in the $[-1,1]$ element natural-coordinate domain.
Figure~\ref{fig:N4_lsfe} shows representative LSFE basis functions for  
fourth- and eighth-order elements.  Note that nodes are clustered near
element endpoints.
In the present implementation, weak-form integrals are evaluated with
$p$-point reduced Gauss quadrature.

\begin{figure}
    \centering
    \psfrag{x}[][]{$\xi$}
   \subfloat[$p=4$]{
   \includegraphics[width=0.3\textwidth,clip=true]{\directory N4.eps}}
   \subfloat[$p=8$]{
   \includegraphics[width=0.3\textwidth,clip=true]{\directory N8.eps}}
    \caption{Representative $p+1$ Lagrangian-interpolant shape functions in
the element natural coordinates for
(a) fourth- and (b) eighth-order LSFEs, where nodes are located at the
Gauss-Lobatto-Legendre points.}
    \label{fig:N4_lsfe}
\end{figure}

The geometrically exact beam theory has been implemented with LSFEs in a
code called  as BeamDyn, which is a module of the
the FAST CAE tool for wind turbine analysis. The system of nonlinear
equations in Eqs.~\eqref{GovernGEBT-1} and \eqref{GovernGEBT-2} are solved
using the Newton-Raphson method with the linearized form in
Eq.~\eqref{LinearizedEqn}. 
In the present
implementation, an energy-like stopping criterion has been chosen, which is calculated as
\begin{equation}
    \label{StoppingCriterion}
    \| \Delta \mathbf{U}^{(i)T} \left( \fourIdx{t+\Delta t}{}{}{}{\mathbf{R}} -  \fourIdx{t+\Delta t}{}{(i-1)}{}{\mathbf{F}}  \right) \| \leq \| \epsilon_E \left( \Delta \mathbf{U}^{(1)T} \left( \fourIdx{t+\Delta t}{}{}{}{\mathbf{R}} - \fourIdx{t}{}{}{}{\mathbf{F}} \right) \right) \|
\end{equation}
where $\|\cdot\|$ denotes the Euclidean norm, $\Delta \mathbf{U}$ is the
incremental displacement vector, $\mathbf{R}$ is the vector of externally
applied nodal point loads, $\mathbf{F}$ is the vector of nodal point forces
corresponding to the internal element stresses, and $\epsilon_E$ is the
preset energy tolerance. The superscript on the left side of a variable
denotes the time-step number (in a dynamic analysis), while the one on the
right side denotes the Newton-Raphson iteration number. As pointed out by
Bathe and Cimento\cite{Bathe-Cimento:1980}, this criterion provides
a measure of when both the displacements and the forces are near their
equilibrium values. Time integration is performed using the
generalized-$\alpha$ scheme in BeamDyn, which is an unconditionally stable
(for linear systems),
second-order accurate algorithm.  The scheme allows for users to choose
integration parameters that introduce high-frequency numerical dissipation.
More details
regarding the generalized-$\alpha$ method can be found in
Refs.\cite{Chung-Hulbert:1993,Bauchau:2010}. 

\mas{Qi: list the $\alpha$-method parameters used in your simulations}

