%&LaTeX
\section{Legendre Spectral Finite Elements}
The displacement fields in a element are interpolated as
\begin{align}
    \label{InterpolateDisp}
    \vec{u}(s) &= h^k(s) \vec{\hat{u}}^k \\
    \label{InterpolateDispp}
    \vec{u}^\prime(s) &= h^{k\prime}(s) \vec{\hat{u}}^k
\end{align}
where $h^k(s)$ is the
Lagrangian-interpolan shape function of node $k$, $k=1,2,...,n+1$, where $n$
is the polynomial order of of the basis functions, and $\vec{\hat{u}}^k$ is
the $k^{th}$ nodal value.
However, as discussed in Ref~\cite{Bauchau-etal:2009}, the three-dimensional rotation field cannot be
simply interpolated as the displacement field in the form of
\begin{align}
    \label{InterpolateRot}
    \vec{c}(s) &= h^k(s) \vec{\hat{c}}^k \\
    \label{InterpolateRotp}
    \vec{c}^\prime(s) &= h^{k \prime}(s) \vec{\hat{c}}^k 
\end{align}    
where $\vec{c}$ is the rotation field in a element and $\vec{\hat{c}}^k$ is
the nodal value at the $k^{th}$ node, for three reasons: 1) rotations do not
form a linear space so that they must be  ``composed'' instead of added; 2)
a rescaling operation is needed to eliminate the singularity existing in the
vectorial rotation parameters; 3) the rotation field lacks objectivity,
which, as
defined by Crisfield and Jeleni\'c\cite{Crisfield1999}, refers to the
invariance of strain measures computed through interpolation to the addition
of a rigid-body motion. Therefore, we adopt the more robust interpolation
approach proposed by Crisfield and Jeleni\'c \cite{Crisfield1999} to deal
with the finite rotations. Our approach is described as follows
\begin{description}
    \item[Step 1:] Compute the nodal relative rotations, $\vec{\hat{r}}^k$ by removing the rigid body rotation, $\vec{\hat{c}}^1$, from the finite rotation at each node, $\vec{\hat{r}}^k = \vec{\hat{c}}^{1-} \oplus \vec{\hat{c}}^k$.
    \item[Step 2:] Interpolate the relative rotation field: $\vec{r}(s) = h^k(s) \vec{\hat{r}}^k$ and $\vec{r}^\prime(s) = h^{k \prime}(s) \vec{\hat{r}}^k$. Find the curvature field $\vec{\kappa}(s) = \tens{R}(\vec{\hat{c}}^1) \tens{H}(\vec{r}) \vec{r}^\prime$.
    \item[Step 3:] Restore the rigid body rotation removed in Step 1: $\vec{c}(s) = \vec{\hat{c}}^1 \oplus \vec{r}(s)$.
\end{description} 
where $H$ is the tangent tensor that relates the curvature vector $k$ and rotation vector $p$ as
\begin{equation}
    \label{Tensor}
    \vec{k} = \tens{H}~ \vec{p}^\prime
\end{equation}
In the LSFE approach, shape functions (e.g., those composing $\tens{N}$) are
$n^{th}$-order Lagrangian interpolants, where nodes are located at the $n+1$
GLL-quadrature points in the $[-1,1]$ element natural-coordinate domain.
\textcolor{red}{Need more work here: a figure shows some LS elements (non-evenly placed internal nodes) and a short discussion of its advantages.}
The present implementation  carries out the integration in the space domain by using $n$ point reduced Gauss quadrature with $n$ the order of the interpolation polynomials. 

The geometrically exact beam theory introduced above has been implemented using Legendre spectral finite element, known as BeamDyn, as a module functioning in the FAST modularization framework. The system of nonlinear equations in Eq.~\eqref{GovernGEBT-1} and \eqref{GovernGEBT-2} are solved using Newton-Raphson method in the linearized form in Eq.~\eqref{LinearizedEqn} at each iteration for corrections to the nodal displacements and rotations until convergence is reached. In the present implementation, a energy-like stopping criterion has been chosen, which is calculated as
\begin{equation}
    \label{StoppingCriterion}
    \| \Delta \mathbf{U}^{(i)T} \left( \fourIdx{t+\Delta t}{}{}{}{\mathbf{R}} -  \fourIdx{t+\Delta t}{}{(i-1)}{}{\mathbf{F}}  \right) \| \leq \| \epsilon_E \left( \Delta \mathbf{U}^{(1)T} \left( \fourIdx{t+\Delta t}{}{}{}{\mathbf{R}} - \fourIdx{t}{}{}{}{\mathbf{F}} \right) \right) \|
\end{equation}
where $\|\bullet\|$ denotes the Euclidean norm, $\Delta \mathbf{U}$ is the incremental displacement vector, $\mathbf{R}$ is the vector of externally applied nodal point loads, $\mathbf{F}$ is the vector of nodal point forces corresponding to the internal element stresses, and $\epsilon_E$ is the preset energy tolerance. As pointed out by Bathe and Cimento in Ref.\cite{Bathe-Cimento:1980}, this criterion provides information of when both the displacements and the forces are near their equilibrium values. Time integration is performed using generalized-$\alpha$ scheme in BeamDyn, which is a unconditionally stable, second-order accurate algorithm. The users can choose proper parameters to achieve high frequency numerical dissipation in this scheme. More details regarding the generalized-$\alpha$ method can be found in Ref.\cite{Chung-Hulbert:1993,Bauchau:2010}. 


