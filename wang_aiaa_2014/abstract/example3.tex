%&LaTeX
\subsection{Example 3: Dynamic analysis of a composite beam under sinusoidal force at the tip}

The last example is a transient analysis of a composite beam with boxed
cross-section; the beam has the same
geometry and boundary conditions as that of the previous example. The mass
sectional properties are given by VABS \cite{Yu-etal:2002,Wang-Yu:2012} as
\begin{equation}
    \label{E3Mass}
    M^* = 10^{-2} \times \begin{bmatrix}
	8.538 & 0     & 0     & 0      & 0      & 0      \\
	0       & 8.538 & 0     & 0      & 0      & 0      \\
	0       & 0     & 8.538 & 0      & 0      & 0      \\
	0       & 0     & 0     & 1.4433  & 0  & 01 \\
	0       & 0     & 0     & 0  & 0.40972  &0 \\
	0       & 0     & 0     & 0 & 0 & 1.0336
\end{bmatrix}    
\end{equation}
The units associated with the mass matrix values are $M_{ii}^*$ (lb s$^2$/in$^2$) and $M_{i+3,i+3}^*$ (lb s$^2$) for $i = 1,2,3$. Similarly to the $C^*$ matrix in Example 1, the sectional mass matrix $\tens{\mathcal{M}}$ resolved in the inertial basis can be obtained by $\tens{\mathcal{M}} = (\tens{R}~\tens{R}_0) \tens{M}^*(\tens{R}~\tens{R}_0)^T$. The beam is divided into two $5^{th}$-order elements in the current calculation and a sinusoidal point dead force is applied at the free tip in the $x_3$ direction given as
\begin{equation}
    \label{E3AppliedForce}
    P_3 = A_F~\text{sin}(\omega_F~t)
\end{equation}
where $A_F = 1.0 \times 10^2$ lbs and $\omega_F = 10$ rad/s (see
Figure~\ref{E3SinForce}). 
The spectral radius $\rho_\infty$ is set to $0.0$ in the time integrator so that high frequency numerical dissipation can be achieved. The tip displacement and rotation histories of the beam are plotted in
Figure~\ref{E3U}, where the time step was $0.005$ s. Note that all of the components, including three displacements and three rotations, are non-zero due to the elastic-coupling effects. The time histories of the stress resultants at the root of the beam are given in Figure~\ref{E3F}.
\begin{figure}
    \centering
    \psfrag{Time}{blah}
    \includegraphics[width=3.0in]{\directory E3SinForce.eps}
    \caption{The applied sinusoidal vertical force at the tip in Example 3 .}
    \label{E3SinForce}
\end{figure}

\begin{figure}
    \centering
    \begin{tabular}{c}
    \subfloat[$u_1$]{\label{E3U:u1}\includegraphics[width=3.0 in]{\directory  E3Tipu1.eps}} \qquad
\subfloat[$u_2$]{\label{E3U:u2}\includegraphics[width=3.0in]{\directory  E3Tipu2.eps}} \\
\subfloat[$u_3$]{\label{E3U:u3}\includegraphics[width=3.0 in]{\directory  E3Tipu3.eps}} \qquad
\subfloat[$p_1$]{\label{E3U:p1}\includegraphics[width=3.0 in]{\directory  E3Tipp1.eps}} \\
\subfloat[$p_2$]{\label{E3U:p2}\includegraphics[width=3.0 in]{\directory  E3Tipp2.eps}} \qquad
\subfloat[$p_3$]{\label{E3U:p3}\includegraphics[width=3.0 in]{\directory  E3Tipp3.eps}} \\
\end{tabular}
\caption{Tip displacement and rotation histories of a composite beam under vertical load.}
\label{E3U}
\end{figure} 

\begin{figure}
    \centering
    \begin{tabular}{c}
    \subfloat[$F_1$]{\label{E3F:F1}\includegraphics[width=3.0 in]{\directory  E3RootF1.eps}} \qquad
\subfloat[$F_2$]{\label{E3F:F2}\includegraphics[width=3.0in]{\directory  E3RootF2.eps}} \\
\subfloat[$F_3$]{\label{E3F:F3}\includegraphics[width=3.0 in]{\directory E3RootF3.eps}} \qquad
\subfloat[$M_1$]{\label{E3F:M1}\includegraphics[width=3.0 in]{\directory  E3RootM1.eps}} \\
\subfloat[$M_2$]{\label{E3F:M2}\includegraphics[width=3.0 in]{\directory  E3RootM2.eps}} \qquad
\subfloat[$M_3$]{\label{E3F:M3}\includegraphics[width=3.0 in]{\directory  E3RootM3.eps}} \\
\end{tabular}
\caption{Stress resultant time histories at the root of a composite beam.}
\label{E3F}
\end{figure} 
Finally, we examine here the convergence rates of the LSFEs and conventional
quadratic elements (in Dymore). Figure~\ref{E3Conv} shows normalized root-mean-square (RMS) error of the numerical solutions for the displacement $u_1$ at the free tip over the time interval $0 \leq t  \leq 4$. Normalized RMS error for $n_{max}$ numerical response values $u_1^n$, where $u_1^n \approx u_1(t^n)$, was calculated as
\begin{equation}
    \label{RMS}
    \varepsilon_{\mathrm{RMS}}(u_1) = \sqrt{\frac{\sum_{k=0}^{n_{max}} \left[ u_1^k - u_b(t^k) \right]^2}{\sum_{k=0}^{n_{max}} \left[ u_b(t^k) \right]^2}}
\end{equation}
where $u_b(t)$ is the benchmark solution; here $u_b(t)$ is a highly resolved
numerical solution obtained by BeamDyn with one $20^{th}$-order element and
the time increment was $\Delta t_b = 1.0 \times 10^{-4}$ s. Two time-increment sizes
are examined in the test calculations: $\Delta t_1 = 5.0 \times 10^{-3}$ s and
$\Delta t_2 = \frac{\Delta t_1}{2}$. The following observations can be made from Figure~\ref{E3Conv}:
\begin{itemize}

    \item For a fixed $\Delta t$, both Dymore (QFEs) and BeamDyn (LSFEs)
converge with spatial refinement to the same error level. BeamDyn is
converged with only five nodes, whereas Dymore requires at least nine nodes.

    \item  The converged error levels are due exclusively to
time-discretization error.  We note that the converged error for $\Delta t_2
= \Delta t_1/2$ is one-fourth that for $\Delta t_1$, which is expected for our
second-order-accurate time integrator.  

%The non-zero errors are in the time integration scheme. By reducing time
%increment step $\Delta t$ by 2 ($\Delta t_2 = \frac{\Delta t_1}{2}$), the
%error is reduced by $4$ ($\varepsilon_2 = \frac{\varepsilon_1}{4}$), which
%is expected for a second order accurate time integrator.

%    \item The convergence rate of LSFE in the space domain is exponential
%    as %expected, which is much faster than the conventional quadratic finite elements.
\end{itemize}  

%It can be observed that \textcolor{red}{2 observations:1 error due to time integrator; 2 better convergence rate of LSFE.}
\begin{figure}
    \centering
    \includegraphics[width = 3.0 in]{\directory E3u1Conv.eps}
    \caption{Normalized RMS error of tip displacement $u_1$ histories over
$0 \leq t \leq 4$ as a function of number of nodes as calculated by BeamDyn
(LSFEs) and Dymore (QFEs).}
    \label{E3Conv}
\end{figure}
 
